\section{Introduction}

Fiber-reinforced composite materials have gained increasing attention due to
their superior mechanical performance in stiffness, strength, and specific
gravity of fibers over conventional materials. Laminated composite material
takes advantage of fiber-reinforced composite material, and finds wide
application in a variety of applications, which include electronic packaging,
sports equipment, homebuilding, medical prosthetic devices, high-performance
military structures, etc. The mechanical properties of composite laminated are
determined by stacking sequence, ply thickness, fiber orientation, and material
for each ply. Strength ratio\cite{todoroki1998stacking,liu2000permutation,sivakumar1998optimum,walker2003technique,lin2004stacking,kang2005minimum,murugan2007target,akbulut2008optimum} is a critical
index to predict the performance of a laminated composite material. There are
two approaches for solving this problem: analytical methods, such as classical
lamination theory(CLT); data-driven methods, such as artificial neural networks
(ANN).

The analytical approach involves a two-step procedure to obtain strength ratio:
first, develop the stress and strain relationship among in-plane loading using
classical lamination theory based on a knowledge of the composite laminate
properties of the individual layers and the laminate geometry; then calculate
the strength ratio according to associated failure criteria, such as Tsai-Wu
failure criterion, based on the above-obtained stress and strain relationship.
However, the use of CLT needs intensive computation since it involves massive
matrix multiplication and integration operation.

The other approach to this problem is using an artificial neural network, which
is a data-driven method, instead of an analytical method. ANN, heavily inspired
by biology and psychology, is a reliable tool instead of a complicated
mathematical model, which can accelerate the calculation process and reduce the
computation cost. It has been widely used to solve various practical engineering
problems in applications\cite{YAN2020108014,MENTGES2021108736}, such as pattern recognition, nonlinear
regression, data mining, clustering, prediction, etc. Evolutionary artificial
neural networks are a subclass of artificial neural networks, in which
evolutionary algorithms are introduced to design the topology of an ANN. For an
artificial neural network, the number of layers, the connection between neurons,
the activation functions used in every neuron, etc., are critical components to
its performance. The design of an ANN can be treated as an optimization
procedure of discrete variables, which can be solved by a genetic algorithm(GA).
It is claimed that the combinations of artificial neural networks and
evolutionary algorithm\cite{lobo2007parameter} can significantly improve the performance of
intelligent systems than that rely on ANNs or evolutionary algorithms alone.

GA, inspired by Darwin’s principle of survival of the fittest, is widely adopted
to obtain the global optimal for discrete optimization problems. The techniques
used in this algorithm, such as selection, crossover, mutation, are derived from
natural selection, and individuals with better fitness get more chances to
breed. Therefore, GA can be integrated into the design of ANN, in which encoding
the information of an artificial neural network into a chromosome
\cite{liu1996evolutionary,rodzin2016neuroevolution}.

The rest of this paper is organized as the following: section II introduces the
CLT and the failure criteria, which is used to check whether the composite
material fails or not in the present study; section III covers the design of an
artificial neural network for a function approximation; section IV reviews the
use of the genetic algorithm in the design of neural network architecture, and
the techniques of parameters optimization during the training process; section V
presents the result of the numerical experiments in different cases; in the
conclusion part, we present and discuss the experiment results.





\section{Result and Discussion}

In this work, we propose to use an artificial neural network as an alternative
way to compute the strength ratio of composite material instead of a two-step
procedure, based on classical lamination and failure theory.
Fig. \ref{fig:ga_nn} shows the changes of the fitness and error during the
evolution procedure. The fitness is obtained through the performance estimation
technique of an artificial neural network. As shown in this figure, fitness
grows during the initial stage; then, it slowly converges as generation
proceeds. It implies genetic algorithm can find a better artificial neural
network with the evolution of the number of neurons in the hidden layer,
connection relationship, activation functions, and connection weights.

\begin{figure}[!tb]
	\includegraphics[width=0.9\linewidth]{./fig/result_ga_ann.png}
	\caption{Fitness and averaged sum-of-squares errors of the pre-trained artificial neural network as generations proceed.}
	\label{fig:ga_nn}
\end{figure}

\begin{figure}[!tb]
	\centering
	\def\svgwidth{\columnwidth}
	\import{fig/}{post_train.pdf_tex}
	\caption{The illustration of the behaviour of fitness on the training dataset during the training session.}
	\label{fig:final_train}
\end{figure}


Fig. \ref{fig:final_train} shows the rest training of the artificial neural
network obtained from the GA, which is a pre-trained ANN. Continue train it with
a standard gradient-based descent algorithm until the error converges.
The target neural network converges rapidly at first, and further training
doesn’t reduce the error efficiently. Then, this artificial neural network is
used to predict the strength ratio of laminated composite material.

\begin{table}[!tb]
	\centering
	\caption{ANN predictions of the Tsai-wu and MS strength ratio with the
	numberical results obtained by CLT.}
	\label{tab:simu}
	\begin{adjustbox}{width=0.5\textwidth}
	\begin{tabular}{cccc|cc|cc}
		\toprule
		\multicolumn{4}{c}{\textbf{Input}} &  \multicolumn{4}{c}{\textbf{Output}} \\
		\midrule
		Load  &  \makecell{Laminate \\ Structure }  & \makecell{Material \\ Property} & \makecell{Failure \\  Property}  &
		\multicolumn{2}{c}{ \makecell {CLT \\MS  Tsai-Wu}} & \multicolumn{2}{c}{ \makecell {ANN \\MS  Tsai-Wu}}\\
		\midrule
		-10,40,20  &  26,-26,168,1.27 & 116.6,7.67,0.27,4.17 & 2062.0,1701.0,70,240,105 & 0.342 & 0.476 & 0.351 & 0.492 \\
		20,-70,-30 &  10,-10,196,1.27 & 181.0,10.3,0.28,7.17 & 1500.0,1500.0,40,246,68  & 0.653 & 0.489 & 0.612 & 0.445 \\ 
		60,-20,0   &  82 -82,128,1.27 & 181.0,10.3,0.28,7.17 & 1500.0,1500.0,40,246,68  & 1.663 & 0.112 & 1.673 & 0.189 \\
		\bottomrule
	\end{tabular}
	\end{adjustbox}
\end{table}

To present the evaluation result of the ANN straightforwardly, several
experiment results from the validation dataset are displayed in Table
\ref{tab:simu}, which are randomly selected.
 Comparing the strength ratio outputs based on CLT and ANN from Table
 \ref{tab:simu}, it is shown that the calculation of strength ratio can be
 achieved using a two-layer neural network, without the intensive computation of
 matrix multiplication.




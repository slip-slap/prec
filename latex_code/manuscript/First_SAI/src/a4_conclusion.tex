\section{Conclusion}
In this paper, an evolutionary artificial neural network model was developed to
predict the strength ratio of laminated composite material under in-plane
loading. We review the use of genetic algorithms and artificial neural networks
as an alternative approach for calculating the strength ratio of an angle ply
laminate under in-plane loading. Traditionally, it is obtained through CLT and
corresponding failure criteria, such as Maximum Stress theory and Tsai-Wu
failure theory.

The main contribution of this work is as follows: 1) propose a two-layer diagram
model for designing a sophisticated neural network in simulating the calculation
of strength ratio, and use a genetic algorithm to explore the search space. 2)
suggest an efficient method to compute the strength ratio instead of adopting
the two-step procedure based on classical lamination theory and related failure
criteria. Compared with experimentally obtained data, it is demonstrated that
ANN is an efficient and simple tool to compute the strength ratio, instead of
the complex analytical mathematical model. Our findings underline the practical
applicability of ANN on the analysis of composite material.

There are more improvements we can make over the search strategy and application
in the area of laminated composite material. The future work is to develop a
more sophisticated ANN, which not only can predict the properties for angle ply
laminate, but also the other type of laminated composite material.


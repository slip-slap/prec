\section{Result}
In present study, the T300/5308 graphite/epoxy materials were used in the
lay-up sequence optimization, and it's properties as shown in
table.\ref{tab:T300/5308}. Two constraints are imposed on the composite
laminates which are the safety factors $SF_{MS}$ , and safety factor $SF_{TW}$,
and the threshold values for both of them is 1. The constraint values of an
individual are $CV_1$ and $CV_2$. So the mutation vector here is a two
dimensional vector $[1 - CV_1, 1 - CV_2 ]$, and the coefficient of length
mutation $C_l$ and angle mutation $C_a$, respectively, chosen here is 20 and 10. 

To verify the reliablity of proposed method, two conditions are concerned: the
first is only two distinct fiber orientation angles in the composite material;
the second involves three distinct ply angles within it. In each situation,
we present the variation process of revelvant indexes, such as the fitness,
strength ratio, and angle. Then, the optimum lay-ups is obtained for various
loading cases.

\begin{figure}[!htb]
	\centering
		\begin{subfigure}[b]{0.8\linewidth}
			\includegraphics[width=\linewidth]{2020-11-10-pre-image/two_distinct_angle_fitness_and_sr.png}
		\end{subfigure}

		\begin{subfigure}[b]{0.8\linewidth}
			\includegraphics[width=\linewidth]{2020-11-10-pre-image/two_distinct_angle_angle_change.png}
		\end{subfigure}

		\begin{subfigure}[b]{0.8\linewidth}
			\includegraphics[width=\linewidth]{2020-11-10-pre-image/two_distinct_angler_number_change.png}
		\end{subfigure}
	\caption{Two distinct angles}
	\label{fig:two_angles}
\end{figure}


\begin{table*}
	\normalsize
\caption{The optimum lay-ups using two distinct fiber angles under various biaxial loading cases}
\label{tab:two_distinct_angle}
\centering
\begin{tabular}{clccc}
	\toprule
	\textbf{Loading} $N_{x}/N_{y}/N_{xy}$ \textbf{(MPa m)}   &
	\makecell{\textbf{Optimum lay-up } \\ \textbf{sequences}  }                        &
	\textbf{Laminate thickness} &  \makecell{\textbf{Safety factor } \\
	\textbf{for Tsai-wu}}  &
	\makecell{\textbf{Safety factor } \\ \textbf{for  maximum stress}}
	 \\
	\midrule
	10/5/0                                         &  $[33_{29}/\text{-}39_{25}/\bar{\text{-}39}]_s$            &     109               &  1.0074      &  1.0246  \\
	20/5/0                                         &  $[33_{22}/\text{-}31_{24}]_s$                             &     92               &  1.0055       &  1.2065    \\
	40/5/0                                         &  $[29_{18}/\text{-}21_{23}/\bar{\text{-}21}]_s$            &     83               &  1.0034       &  1.7350   \\
	80/5/0                                         &  $[\text{-}20_{27}/21_{25}/\bar{25}]_s$                    &     105               &  1.0029      &  1.2063    \\
	120/5/0                                         &  $[\text{-}18_{34}/17_{36}]_s$                            &     140               &  1.0000      &  1.0898    \\
	\bottomrule
\end{tabular}
\end{table*}

Figure \ref{fig:two_angles} (a) shows how the optimal individual's fitness and
strength ratio evolves during the GA process. If the smaller strength ratio
fullfils the constraint, this laminate must satisfy all the constraints, for
simplicity, only the smaller strength ratio is presented in the figure. The
method to chose optimal individual considering two following situations, if no
individual in the current population meets constraint, the one with biggest
fitness is selected as the optimal individual; if there are one or multiple
individuals fullfils requirement, the one with smallest fitness is chosen which
means the smallest one has the biggest priority.  Figure \ref{fig:two_angles}
(b) shows how the two distinct fiber orientation changes at the same time, and
Figure \ref{fig:two_angles} (c) displays how the number of each angles change.

At the beginning of this GA process, the fitness curves increases very quickly,
becasue individual's two strength ratios are very small, so the difference
between the individual's fitness and the imposed constraint threshold is a big
positive number, so the range of mutaion length is from 0 to $C_l(CT_0 - CV_0 +
CT_1 - CV_1)/2$.  The length of individual increases by n, which is random
number between 0 and $C_l(CT_0 - CV_0 + CT_1 - CV_1)/2$ . As can be seen from
Figure \ref{fig:two_angles} (a), both of optimal individual's fitness and
strength ratio increases very quickly.  The range of angle mutation is from 0 to
$C_a(CT_0 - CV_0 + CT_1 - CV_1)/2$, and the number of each angle also changes
violently. During this stage, increasing individual's length playing a major
role in increasing individual's fitness.

After a couple of generations, the optimal individual's fitness get bigger, and
the difference between individual's fitness and constraint threshold get
smaller. The range of mutation length turn smaller. At
this stage, simply increase the individual's length doesn't make much difference
in improve individual's fitnees, and a better composite laminates lay-up can
dramaticly change the optimal individual's fitness. That's why the fitness curve
oscillated violently in this stage.  At the same time, the strength ratio curve
kept growing smoothly. But the growing speed got more smaller.

When GA comes to its last phase, GA finds individuals that meet all the
constraints.  Now the optimal individual's fitness is greater than the safety
factor. The range of mutation length is from $C_l(CT_0 - CV_0 + CT_1 - CV_1)/2$
to 0. It means individuals need to decrease it's length and improve its internal
structure to meet the constraint. That's why the fitness of optimal individual
kept decreaing, however, the strength ratio curve still is greater then safety
factor.

\begin{figure}[!t]
	\centering
		\begin{subfigure}[b]{0.8\linewidth}
			\includegraphics[width=\linewidth]{2020-11-10-pre-image/Three_distinct_angles_fitness_and_sr.png}
		\end{subfigure}

		\begin{subfigure}[b]{0.8\linewidth}
			\includegraphics[width=\linewidth]{2020-11-10-pre-image/three_distinct_angles_angle_change.png}
		\end{subfigure}

		\begin{subfigure}[b]{0.8\linewidth}
			\includegraphics[width=\linewidth]{2020-11-10-pre-image/three_distinct_angle_number_of_angle.png}
		\end{subfigure}
	\caption{Three distinct angles}
	\label{fig:three_angles}
\end{figure}


\begin{table*}
\normalsize
\caption{The optimum lay-ups using three distinct fiber angles under various biaxial loading cases}
\label{T300/5308 material properties}
\centering
\begin{tabular}{clccc}
	\toprule
	\textbf{Loading} $N_{x}/N_{y}/N_{xy}$ \textbf{(MPa m)}   &
	\makecell{\textbf{Optimum lay-up } \\ \textbf{sequences}  }                        &
	\textbf{Laminate thickness} &  \makecell{\textbf{Safety factor } \\
	\textbf{for Tsai-wu}}  &
	\makecell{\textbf{Safety factor } \\ \textbf{for  maximum stress}}
	 \\
	\midrule
	10/5/0                       &  $[37_{27}/\text{-}38_{27}/\text{-}5]_s$            &     110      &  1.0023 & 1.0216\\
	20/5/0                       &  $[34_{24}/\text{-}32_{14}/\text{-}28_{11}]_s$      &     98       &  1.0237 & 1.2089 \\
	40/5/0                       &  $[21_{28}/\text{-}32_{19}/2_3]_s$                  &     100      &  1.0617 & 1.7076\\
	80/5/0                       &  $[\text{-}19_{24}/20_{27}/\text{-}{17}_{16}/\bar{\text{-}17}]_s$  &  109      &  1.0056 & 1.2093 \\
	120/5/0                      &  $[\text{-}19_{33}/12_{13}/16_{28}]_s$              &     148      &  1.0105 &  1.1014\\
	\bottomrule
\end{tabular}
\end{table*}

\begin{table*}
	\normalsize
\caption{Comparison with the results of DSA}
\label{tab:comparision}
\centering
\begin{tabular}{c|cccc|lccc}
	\toprule
	\textbf{Loading}	    & \multicolumn {4}{c}{\textbf{Akbulut and Sonmez's\cite{akbulut2008optimum} Study}}   & \multicolumn {4}{c}{\textbf{Present Study}}\\
	\midrule
	 $N_{x}/N_{y}/N_{xy}$   & Optimum lay-up			        & laminate  & TW & MS   & Optimum lay-up & laminate  & TW & MS \\
	  (MPa m)	            & sequences					        & thickness &    &      & sequences	     & thickness &    &    \\
	\midrule
	  10/5/0                 &  $[37_{27}/\text{-}37_{27}]_s$     &  108      &  1.0068  &  1.0277 & $[33_{29}/\text{-}39_{25}/\bar{\text{-}39}]_s$     &     109      &  1.0074      &  1.0246  \\
	  20/5/0                 &  $[31_{23}/\text{-}31_{23}]_s$     &  92       &  1.0208  &  1.1985 & $[33_{22}/\text{-}31_{24}]_s$                      &     92      &  1.0055       &  1.2065    \\
	  40/5/0                 &  $[26_{20}/\text{-}26_{20}]_s$     &  80       &  1.0190  &  1.5381 & $[29_{18}/\text{-}21_{23}/\bar{\text{-}21}]_s$     &     83      &  1.0034       &  1.7350   \\
	  80/5/0                 &  $[21_{25}/\text{-}19_{28}]_s$     &  106      &  1.0113  &  1.2213 & $[\text{-}20_{27}/21_{25}/\bar{25}]_s$             &     105      &  1.0029      &  1.2063    \\
	  120/5/0                &  $[17_{35}/\text{-}17_{35}]_s$     &  140      &  1.0030  &  1.0950 & $[\text{-}18_{34}/17_{36}]_s$                     &     140      &  1.0000      &  1.0898     \\
	\bottomrule
\end{tabular}
\end{table*}

Table.\ref{tab:comparision} shows the comparison with the result obtained by
direct search simulated annealing(DSA) algorithm which is proposed by Akbulut
and Sonmez\cite{akbulut2008optimum}.  Both of variant GA and DSA are able to
find feasible solution, but when loading is $N_x=80$, $N_y=5$ MPa m, variant GA
gets a better solution than DSA. In the case that loadings are $N_x=20$, $N_y=5$ MPa m,
and $N_x=120$, $N_y=5$ MPa m,  the proposed GA offers an alternative. Compared
with DSA method, the last advantage of variant GA is the number of layers
doesn't have to be an even number.


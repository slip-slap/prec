\section{Methodology}
For an angle ply laminate, given the laminate's lay-up, material properties,
in-plane loading, etc., we can compute its strength ratio based on Tsai-Wu
failure theory or maximum stress theory. To model this function, we propose an
ANN framework shown in Fig.\ref{fig:evolution}, which derives from the previous
two-layer model. There are sixteen inputs of this ANN, which are in-plane
loading $N_x$, $N_y$, and $N_{xy}$; design parameters of a laminate, two 
fiber orientation $\theta_1$ and $\theta_2$, ply thickness $t$, total
number of plies $N$; 
five engineering constants of composite materials,
$E_1$, $E_2$, $G_{12}$, and $v_{12}$; five strength parameters of a
unidirectional lamina.  Two outputs are strength ratio accoring to MS theory
and strength ratio according to Tsai-Wu theory.

\begin{figure*}
\centering
\begin{tikzpicture}
[ p/.style={ draw=none, fill=none, }, remember picture, 
  net/.style={ matrix of nodes, nodes={ draw, circle, inner sep=7.5pt },
  nodes in empty cells,
  column sep=-10.5pt,
  row sep=0.8cm
  }
]
%\draw[help lines] (-3cm,-6cm) grid (6cm,3cm);
\matrix[net] (mat)
{
	  & |[p]| &  & |[p]| &  & |[p]| &  & |[p]| &  & |[p]| &  & |[p]| &  & |[p]| &  & |[p]| &  &
	    |[p]| &  & |[p]| &  & |[p]| &  & |[p]| &  & |[p]| &  & |[p]| &  & |[p]| &  & |[p]|    \\
 |[p]| & |[p]| & |[p]| &  |[p]| &        & |[p]| & |[p]| & |[p]| &|[p]| &       & |[p]| &  |[p]| & |[p]| &
 |[p]| &       & |[p]| &  |[p]| &  |[p]| & |[p]| & |[p]| &       &|[p]| & |[p]| & |[p]| & |[p]|
	   & |[p]| &       &  |[p]| &  |[p]| & |[p]| & |[p]| & |[p]| &|[p]| \\ 
 |[p]| &  |[p]| & |[p]|  &  |[p]| & |[p]|  &  |[p]| &  |[p]| &  |[p]| & |[p]| & |[p]| & |[p]| &       & |[p]|
	   &  |[p]| & |[p]|  &  |[p]| &        &  |[p]| &  |[p]| &  |[p]| & |[p]| & |[p]| & |[p]| & |[p]| &     |[p]|
	   &  |[p]| & |[p]|  &  |[p]| & |[p]|  &  |[p]| &  |[p]| &  |[p]| \\ 
  };
  \draw[<-] (mat-1-1.north) --  ++(0,1) node {$N_x$};
  \draw[<-] (mat-1-3.north) --  ++(0,1) node {$N_y$};
  \draw[<-] (mat-1-5.north) --  ++(0,1) node {$N_{xy}$};
  \draw[<-] (mat-1-7.north) --  ++(0,1) node {$\theta_1$};
  \draw[<-] (mat-1-9.north) --  ++(0,1) node {$\theta_2$};
  \draw[<-] (mat-1-11.north) --  ++(0,1) node {$t$};
  \draw[<-] (mat-1-13.north) --  ++(0,1) node {$N$};
  \draw[<-] (mat-1-15.north) --  ++(0,1) node {$E_1$};
  \draw[<-] (mat-1-17.north) --  ++(0,1) node {$E_2$};
  \draw[<-] (mat-1-19.north) --  ++(0,1) node {$G_{12}$};
  \draw[<-] (mat-1-21.north) --  ++(0,1) node {$v_{12}$};
  \draw[<-] (mat-1-23.north) --  ++(0,1) node {$\sigma_1^T$};
  \draw[<-] (mat-1-25.north) --  ++(0,1) node {$\sigma_1^C$};
  \draw[<-] (mat-1-27.north) --  ++(0,1) node {$\sigma_2^T$};
  \draw[<-] (mat-1-29.north) --  ++(0,1) node {$\sigma_2^C$};
  \draw[<-] (mat-1-31.north) --  ++(0,1) node {$\tau_{12}$};
  \draw[->] (mat-3-12.south) --  ++(0,-1) node[pos=0.5, swap] {Tsai-Wu};
  \draw[->] (mat-3-17.south) --  ++(0,-1) node[pos=0.5, swap] {MS};
  \node at ($(mat-1-1.west)+(-16pt,0)$) {Input: };
  \node at ($(mat-2-2.west)+(-24pt,0)$) {Hidden:};
  \node at ($(mat-3-2.west)+(-24pt,0)$) {Output:};
  \node at (mat-2-5.base) {$h_1$};
  \node at (mat-2-10.base) {$h_2$};
  \node at (mat-2-15.base) {$...$};
  \node at (mat-2-21.base) {\small{$h_{m-1}$}};
  \node at (mat-2-27.base) {$h_{m}$};
 \foreach \a in {1,3,5,7,9,11,31}{
        \draw[->] (mat-1-\a.south) -- (mat-2-5.north);
     }
 \foreach \a in {5,7,11,19,25,27}{
        \draw[->] (mat-1-\a.south) -- (mat-2-10.north);
     }
 \foreach \a in {1,7,11,17,19,25}{
        \draw[->] (mat-1-\a.south) -- (mat-2-15.north);
     }
 \foreach \a in {5,9,19,21,29}{
        \draw[->] (mat-1-\a.south) -- (mat-2-21.north);
     }
 \foreach \a in {11,15,19,23,27,29,31}{
        \draw[->] (mat-1-\a.south) -- (mat-2-27.north);
     }
 \foreach \c in {5,10,15,21,27}{
    \foreach \d in {12,17}{
 		\draw[->] (mat-2-\c.south) -- (mat-3-\d.north);
	}
 }
\end{tikzpicture}
\caption{Neural Network Model}
\end{figure*}


\begin{figure}[!tb]
	\centering
	\begin{subfigure}[b]{1.0\linewidth}
		\centering
		%\begin{figure}
%\centering
\begin{tikzpicture}
	[ p/.style={ draw=none, fill=none}, 
	  remember picture, 
	  net/.style={ matrix of nodes, nodes={ draw, circle, inner sep=7.5pt },
	  nodes in empty cells,
	  column sep=-10.5pt,
	  row sep=0.8cm,
	  },
	  scale=0.6,
	  every node/.style={}
	]
%\draw[help lines] (-3cm,-6cm) grid (6cm,3cm);
\matrix[net] (mat)
{
		  & |[p]| &  & |[p]| &  & |[p]| &  & |[p]| &  & |[p]| &  & |[p]| &  & |[p]| &  & |[p]| &  &
			|[p]| &  & |[p]| &  & |[p]| &  & |[p]| &  & |[p]| &  & |[p]| &  & |[p]| &  & |[p]|    \\
	 |[p]| & |[p]| & |[p]| &  |[p]| &        & |[p]| & |[p]| & |[p]| &|[p]| &       & |[p]| &  |[p]| & |[p]| &
	 |[p]| &       & |[p]| &  |[p]| &  |[p]| & |[p]| & |[p]| &       &|[p]| & |[p]| & |[p]| & |[p]|
		   & |[p]| &       &  |[p]| &  |[p]| & |[p]| & |[p]| & |[p]| &|[p]| \\ 
	 |[p]| &  |[p]| & |[p]|  &  |[p]| & |[p]|  &  |[p]| &  |[p]| &  |[p]| & |[p]| & |[p]| & |[p]| &       & |[p]|
		   &  |[p]| & |[p]|  &  |[p]| &        &  |[p]| &  |[p]| &  |[p]| & |[p]| & |[p]| & |[p]| & |[p]| &     |[p]|
		   &  |[p]| & |[p]|  &  |[p]| & |[p]|  &  |[p]| &  |[p]| &  |[p]| \\ 
	  };
	  \node at (mat-1-1.base)  {$i_1$};
	  \node at (mat-1-3.base)  {$i_2$};
	  \node at (mat-1-5.base)  {$i_3$};
	  \node at (mat-1-7.base)  {$i_4$};
	  \node at (mat-1-9.base)  {$i_5$};
	  \node at (mat-1-11.base)  {$i_6$};
	  \node at (mat-1-13.base)  {$i_7$};
	  \node at (mat-1-15.base)  {$i_8$};
	  \node at (mat-1-17.base)  {$i_9$};
	  \node at (mat-1-17.base)  {$i_9$};
	  \node at (mat-1-19.base)  {$i_{10}$};
	  \node at (mat-1-21.base)  {$i_{11}$};
	  \node at (mat-1-23.base)  {$i_{12}$};
	  \node at (mat-1-25.base)  {$i_{13}$};
	  \node at (mat-1-27.base)  {$i_{14}$};
	  \node at (mat-1-29.base)  {$i_{15}$};
	  \node at (mat-1-31.base)  {$i_{16}$};

	  \node at (mat-2-5.base)  {$h_1$};
	  \node at (mat-2-10.base) {$h_2$};
	  \node at (mat-2-15.base) {$h_3$};
	  \node at (mat-2-21.base) {$h_4$};
	  \node at (mat-2-27.base) {$h_5$};
		 \foreach \a in {1,3,5,7,9,11,31}{
				\draw[->,dotted] (mat-1-\a.south) -- (mat-2-5.north);
			 }
		 \foreach \a in {5,7,11,13,19,25,27}{
				\draw[->,dotted] (mat-1-\a.south) -- (mat-2-10.north);
			 }
		 \foreach \a in {1,7,11,13,17,19,25}{
				\draw[->,dotted] (mat-1-\a.south) -- (mat-2-15.north);
			 }
		 \foreach \a in {5,9,19,21,29}{
				\draw[->,dotted] (mat-1-\a.south) -- (mat-2-21.north);
			 }
		 \foreach \a in {11,15,19,23,27,29,31}{
				\draw[->,dotted] (mat-1-\a.south) -- (mat-2-27.north);
			 }
		 \foreach \c in {5,10,15,21,27}{
			\foreach \d in {12,17}{
				\draw[->,dotted] (mat-2-\c.south) -- (mat-3-\d.north);
			}
 }
\end{tikzpicture}
%\caption{Parent 1\label{fig:parents1}}
%\end{figure}

		\caption{Parent 1}
	\end{subfigure}
	\newline
	\begin{subfigure}[b]{1.0\linewidth}
		\centering
		\begin{figure*}
\centering
\begin{tikzpicture}
	[ p/.style={ draw=none, fill=none, }, remember picture, 
	  net/.style={ matrix of nodes, nodes={ draw, circle, inner sep=7.5pt },
	  nodes in empty cells,
	  column sep=-10.5pt,
	  row sep=0.8cm
	  }
	]
%\draw[help lines] (-3cm,-6cm) grid (6cm,3cm);
\matrix[net] (mat)
	{
		  & |[p]| &  & |[p]| &  & |[p]| &  & |[p]| &  & |[p]| &  & |[p]| &  & |[p]| &  & |[p]| &  &
			|[p]| &  & |[p]| &  & |[p]| &  & |[p]| &  & |[p]| &  & |[p]| &  & |[p]| &  & |[p]|    \\
	 |[p]| &  |[p]|& |[p]| &        &  |[p]| &       & |[p]| &       &|[p]| &       & |[p]| &   & |[p]| &
		   &  |[p]|&       &  |[p]| &       & |[p]| &       &|[p]| &       & |[p]| &  
		   & |[p]| &       &  |[p]| &  |[p]| & |[p]| & |[p]| & |[p]| &|[p]| & |[p]| \\ 
	 |[p]| &  |[p]| & |[p]|  &  |[p]| & |[p]|  &  |[p]| &  |[p]| &  |[p]| & |[p]| & |[p]| & |[p]| &       & |[p]|
		   &  |[p]| & |[p]|  &  |[p]| &        &  |[p]| &  |[p]| &  |[p]| & |[p]| & |[p]| & |[p]| & |[p]| &     |[p]|
		   &  |[p]| & |[p]|  &  |[p]| & |[p]|  &  |[p]| &  |[p]| &  |[p]| \\ 
	  };
	  \node at (mat-1-1.base)  {$i_1$};
	  \node at (mat-1-3.base)  {$i_2$};
	  \node at (mat-1-5.base)  {$i_3$};
	  \node at (mat-1-7.base)  {$i_4$};
	  \node at (mat-1-9.base)  {$i_5$};
	  \node at (mat-1-11.base)  {$i_6$};
	  \node at (mat-1-13.base)  {$i_7$};
	  \node at (mat-1-15.base)  {$i_8$};
	  \node at (mat-1-17.base)  {$i_9$};
	  \node at (mat-1-17.base)  {$i_9$};
	  \node at (mat-1-19.base)  {$i_{10}$};
	  \node at (mat-1-21.base)  {$i_{11}$};
	  \node at (mat-1-23.base)  {$i_{12}$};
	  \node at (mat-1-25.base)  {$i_{13}$};
	  \node at (mat-1-27.base)  {$i_{14}$};
	  \node at (mat-1-29.base)  {$i_{15}$};
	  \node at (mat-1-31.base)  {$i_{16}$};
	  \node at (mat-2-4.base)  {$h_1$};
	  \node at (mat-2-6.base) {$h_2$};
	  \node at (mat-2-8.base) {$h_32$};
	  \node at (mat-2-10.base) {$h_4$};
	  \node at (mat-2-12.base) {$h_5$};
	  \node at (mat-2-14.base) {$h_6$};
	  \node at (mat-2-16.base) {$h_7$};
	  \node at (mat-2-18.base) {$h_8$};
	  \node at (mat-2-20.base) {$h_9$};
	  \node at (mat-2-22.base) {$h_{10}$};
	  \node at (mat-2-24.base) {$h_{11}$};
	  \node at (mat-2-26.base) {$h_{12}$};
	 \foreach \a in {13,15,17,19,21}{
			\draw[->, dashed] (mat-1-\a.south) -- (mat-2-4.north);
		 }
	 \foreach \a in {1,3,5,7}{
			\draw[->, dashed] (mat-1-\a.south) -- (mat-2-6.north);
		 }
	 \foreach \a in {1,3,5,7,17,19,21,23}{
			\draw[->, dashed] (mat-1-\a.south) -- (mat-2-8.north);
		 }
	 \foreach \a in {5,9,11,13,15}{
			\draw[->, dashed] (mat-1-\a.south) -- (mat-2-10.north);
		 }
	 \foreach \a in {23,27,29,31}{
			\draw[->, dashed] (mat-1-\a.south) -- (mat-2-12.north);
		 }
	 \foreach \a in {11,15,19}{
			\draw[->, dashed] (mat-1-\a.south) -- (mat-2-14.north);
		 }
	 \foreach \a in {27,29,31}{
			\draw[->, dashed] (mat-1-\a.south) -- (mat-2-16.north);
		 }
	 \foreach \a in {11,19,27,31}{
			\draw[->, dashed] (mat-1-\a.south) -- (mat-2-18.north);
		 }
	 \foreach \a in {15,19,23}{
			\draw[->, dashed] (mat-1-\a.south) -- (mat-2-20.north);
		 }
	 \foreach \a in {3,5,7}{
			\draw[->, dashed] (mat-1-\a.south) -- (mat-2-22.north);
		 }
	 \foreach \a in {17,19,21,23}{
			\draw[->, dashed] (mat-1-\a.south) -- (mat-2-24.north);
		 }
	 \foreach \a in {21,23,25,27,29}{
			\draw[->, dashed] (mat-1-\a.south) -- (mat-2-26.north);
		 }
	 \foreach \c in {4,6,8,10,12,14,16,18,20,22,24,26}{
		\foreach \d in {12,17}{
			\draw[->, dashed] (mat-2-\c.south) -- (mat-3-\d.north);
		}
	 }
\end{tikzpicture}
\caption{Parent 2}
\end{figure*}




		\caption{Parent 2}
	\end{subfigure}
	\newline
	\begin{subfigure}[b]{1.0\linewidth}
		\centering
		\begin{figure*}
\centering
\begin{tikzpicture}
[ p/.style={ draw=none, fill=none, }, remember picture, 
  net/.style={ matrix of nodes, nodes={ draw, circle, inner sep=7.5pt },
  nodes in empty cells,
  column sep=-10.5pt,
  row sep=0.8cm
  }
]
%\draw[help lines] (-3cm,-6cm) grid (6cm,3cm);
\matrix[net] (mat)
{
	  & |[p]| &  & |[p]| &  & |[p]| &  & |[p]| &  & |[p]| &  & |[p]| &  & |[p]| &  & |[p]| &  &
	    |[p]| &  & |[p]| &  & |[p]| &  & |[p]| &  & |[p]| &  & |[p]| &  & |[p]| &  & |[p]|    \\
 |[p]| & |[p]| & |[p]|   & |[p]|  &  |[p]| &  |[p]| & |[p]| &|[p]|  &   & |[p]| &      & |[p]| &  &  |[p]| &  &
 |[p]| &       & |[p]| &   & |[p]|   &  & |[p]| &       &|[p]| & |[p]| & |[p]| & |[p]|
	   & |[p]| &  |[p]|   &  |[p]| &  |[p]| & |[p]| & |[p]|  \\ 
 |[p]| &  |[p]| & |[p]|  &  |[p]| & |[p]|  &  |[p]| &  |[p]| &  |[p]| & |[p]| & |[p]| & |[p]| &       & |[p]|
	   &  |[p]| & |[p]|  &  |[p]| &        &  |[p]| &  |[p]| &  |[p]| & |[p]| & |[p]| & |[p]| & |[p]| &     |[p]|
	   &  |[p]| & |[p]|  &  |[p]| & |[p]|  &  |[p]| &  |[p]| &  |[p]| \\ 
  };
  \node at (mat-1-1.base)   {$i_1$};    \node at (mat-1-3.base)   {$i_2$}; 
  \node at (mat-1-5.base)   {$i_3$};    \node at (mat-1-7.base)   {$i_4$}; 
  \node at (mat-1-9.base)   {$i_5$};    \node at (mat-1-11.base)  {$i_6$}; 
  \node at (mat-1-13.base)  {$i_7$};    \node at (mat-1-15.base)  {$i_8$}; 
  \node at (mat-1-17.base)  {$i_9$};    \node at (mat-1-17.base)  {$i_9$};
  \node at (mat-1-19.base)  {$i_{10}$}; \node at (mat-1-21.base)  {$i_{11}$};
  \node at (mat-1-23.base)  {$i_{12}$}; \node at (mat-1-25.base)  {$i_{13}$};
  \node at (mat-1-27.base)  {$i_{14}$}; \node at (mat-1-29.base)  {$i_{15}$};
  \node at (mat-1-31.base)  {$i_{16}$};

  \node at (mat-2-9.base)  {$h_1$};
  \node at (mat-2-11.base) {$h_2$};
  \node at (mat-2-13.base) {$h_3$};
  \node at (mat-2-15.base) {$h_4$};
  \node at (mat-2-17.base) {$h_5$};
  \node at (mat-2-19.base) {$h_6$};
  \node at (mat-2-21.base) {$h_7$};
  \node at (mat-2-23.base) {$h_8$};

 \foreach \a in {1,3,5,7,9,11,31}{
        \draw[->,dotted] (mat-1-\a.south) -- (mat-2-9.north);
     }
 \foreach \a in {5,7,11,13,19,25,27}{
        \draw[->,dotted] (mat-1-\a.south) -- (mat-2-11.north);
     }
 \foreach \a in {27,29,31}{
        \draw[->,dashed] (mat-1-\a.south) -- (mat-2-13.north);
     }
 \foreach \a in {11,19,27,31}{
        \draw[->,dashed] (mat-1-\a.south) -- (mat-2-15.north);
     }
 \foreach \a in {15,19,23}{
        \draw[->,dashed] (mat-1-\a.south) -- (mat-2-17.north);
     }
 \foreach \a in {3,5,7}{
        \draw[->,dashed] (mat-1-\a.south) -- (mat-2-19.north);
     }
 \foreach \a in {17,19,21,23}{
        \draw[->,dashed] (mat-1-\a.south) -- (mat-2-21.north);
     }
 \foreach \a in {21,23,25,27,29}{
        \draw[->,dashed] (mat-1-\a.south) -- (mat-2-23.north);
     }

 \foreach \c in {9,11,13,15,17,19,21,23}{
    \foreach \d in {12,17}{
 		\draw[->] (mat-2-\c.south) -- (mat-3-\d.north);
	}
 }

\end{tikzpicture}
\caption{Child}
\end{figure*}




		\caption{Child}
	\end{subfigure}
	\caption{Examples of three ANN, with (a) and (b) as parent ANNs, and (c) as
		the child of (a) and (b). child c inherits the connection relationship
		part from parent 1 denoted by the darker dashed lines,and the rest from
		parent 2 denoted by the gray dashed line.}
		\label{fig:anns}
\end{figure}
\begin{table*}[!t]
		\centering
		\caption{The binary representation of parent 1, parent 2, and child
			corresponding to Fig .\ref{fig:three_anns}(a), (b) and (c), with
			$i_1,i_2,\cdots, i_{16}$ denote sixteen inputs and $h_1,h_2,
			\cdots, h_{12}$ refer to nodes in the hidden layer. 1 represents an
			edge from the input node to the hidden node, and 0 represents no
			edge from the input nodes to the hidden node.}
		\label{tab:binary-rep}
		\begin{adjustbox}{width=0.9\textwidth}
		\begin{tabular}{cl|cccc cccc cccc cccc | cc}
			\toprule
			Hidden & Nodes  & $i_1$ & $i_2$ & $i_3$ & $i_4$ & $i_5$ & $i_6$ & $i_7$ & $i_8$ & $i_9$ & $i_{10}$ & $i_{11}$ & $i_{12}$ & $i_{13}$ & $i_{14}$ & $i								_{15}$ & $i_{16}$ & f & f\\
			\midrule
			\multirow{5}{*}{P1}		
				   &	$h_1$ & 1  & 1 & 1  & 1  & 1 & 1 & 0 & 0  & 0 & 0 & 0 & 0  & 0 & 0  & 1 & 1 & 0 & 0\\
				&	$h_2$ & 0  & 1 & 1  & 1  & 0 & 0 & 0 & 1  & 0 & 0 & 1 & 1  & 0 & 0  & 0 & 0 & 1 & 1\\
				&	$h_3$ & 1  & 0 & 0  & 1  & 0 & 1 & 1 & 0  & 1 & 1 & 0 & 0  & 1 & 0  & 0 & 0 & 0 & 0\\
				&	$h_4$ & 0  & 0 & 1  & 0  & 1 & 0 & 0 & 0  & 0 & 1 & 0 & 1  & 0 & 0  & 1 & 0 & 0 & 1\\
				&	$h_5$ & 0  & 0 & 0  & 0  & 0 & 1 & 0 & 1  & 0 & 1 & 0 & 1  & 0 & 1  & 1 & 1 & 0 & 1\\
			\midrule
			\multirow{12}{*}{P2}	
				&	$h_1$ & 0  & 0 & 0  & 0  & 0 & 0 & 1 & 1  & 1 & 1	 & 1 & 0  & 0 & 0  & 0 & 0 & 1 & 0\\
				&	$h_2$ & 1  & 1 & 1  & 1  & 0 & 0 & 0 & 0  & 0 & 0	 & 0 & 0  & 0 & 0  & 0 & 0 & 0 & 0\\
				&	$h_3$ & 1  & 1 & 1  & 1  & 0 & 0 & 0 & 0  & 1 & 1	 & 1 & 1  & 0 & 0  & 0 & 0 & 1 & 1\\
				&	$h_4$ & 0  & 0 & 1  & 1  & 1 & 1 & 1 & 0  & 0 & 0	 & 0 & 0  & 0 & 0  & 0 & 0 & 0 & 0\\
				&	$h_5$ & 0  & 0 & 0  & 0  & 0 & 0 & 0 & 0  & 0 & 0	 & 0 & 1  & 0 & 1  & 1 & 1 & 0 & 1\\
				&	$h_6$ & 0  & 0 & 0  & 0  & 0 & 1 & 0 & 1  & 0 & 1	 & 0 & 1  & 0 & 1  & 1 & 1 & 0 & 1\\
				&	$h_7$ & 0  & 0 & 0  & 0  & 0 & 0 & 0 & 0  & 0 & 0	 & 0 & 0  & 0 & 1  & 1 & 1 & 0 & 0\\
				&	$h_8$ & 0  & 0 & 0  & 0  & 0 & 1 & 0 & 0  & 0 & 1	 & 0 & 0  & 0 & 1  & 0 & 1 & 0 & 0\\
				&	$h_9$ & 0  & 0 & 0  & 0  & 0 & 1 & 0 & 1  & 0 & 1	 & 0 & 1  & 0 & 0  & 0 & 0 & 0 & 1\\
				&	$h_{10}$   & 0  & 1 & 1  & 1  & 0 & 0 & 0 & 0  & 1 & 1 & 1 & 1  & 0 & 0  & 0 & 0 & 1 & 1\\
				&	$h_{11}$   & 0  & 0 & 0  & 0  & 0 & 0 & 0 & 0  & 1 & 1 & 1 & 1  & 0 & 0  & 0 & 0 & 1 & 1\\
				&	$h_{12}$   & 0  & 0 & 0  & 0  & 0 & 0 & 0 & 0  & 0 & 0 & 1 & 1  & 1 & 1  & 1 & 0 & 1 & 1\\
			\midrule
			\multirow{8}{*}{Child}	
				   &	$h_1$ & 1  & 1 & 1  & 1  & 1 & 1 & 0 & 0  & 0 & 0 & 0 & 0  & 0 & 0  & 1 & 1 & 0 & 0\\
				&	$h_2$ & 0  & 1 & 1  & 1  & 0 & 0 & 0 & 1  & 0 & 0 & 1 & 1  & 0 & 0  & 0 & 0 & 1 & 1\\
				&	$h_1$ & 0  & 0 & 0  & 0  & 0 & 0 & 1 & 1  & 1 & 1	 & 1 & 0  & 0 & 0  & 0 & 0 & 1 & 0\\
				&	$h_2$ & 1  & 1 & 1  & 1  & 0 & 0 & 0 & 0  & 0 & 0	 & 0 & 0  & 0 & 0  & 0 & 0 & 0 & 0\\
				&	$h_3$ & 1  & 1 & 1  & 1  & 0 & 0 & 0 & 0  & 1 & 1	 & 1 & 1  & 0 & 0  & 0 & 0 & 1 & 1\\
				&	$h_4$ & 0  & 0 & 1  & 1  & 1 & 1 & 1 & 0  & 0 & 0	 & 0 & 0  & 0 & 0  & 0 & 0 & 0 & 0\\
				&	$h_5$ & 0  & 0 & 0  & 0  & 0 & 0 & 0 & 0  & 0 & 0	 & 0 & 1  & 0 & 1  & 1 & 1 & 0 & 1\\
				&	$h_6$ & 0  & 0 & 0  & 0  & 0 & 1 & 0 & 1  & 0 & 1	 & 0 & 1  & 0 & 1  & 1 & 1 & 0 & 1\\
			\bottomrule
			\end{tabular}
		\end{adjustbox}
\end{table*}




The work involved in the evolution process of ANN consists of three parts:
search space, which includes the ANN's topology, transfer function, etc.;
search strategy, which details how to explore the search space; performance
estimation strategy refers to the process of estimating this performance.

\subsection{Search Space}
we propose a GNN framework as shown in Figure \ref{fig:gnn}. The search space
is parametrized by: (i) the number of nodes m(possibly unbounded) in hidden
layer, to narrow down the search space, the assumptions is that m less than n; (ii) the type of
operation every nodes executes, e.g., sigmoid, linear, gaussian. (iii) the
connection relationship between hidden nodes and inputs; (IV) if a connection
exists, the weight value in the connection.

Therefore, evolution in EANN can be divided into four different levels: topology, learning
rules, active functions, and connection weights. For the evolution of toplogy,
the aim is to find an optimal ANN architecture for a specfic problem. The
architecture of a neural network determines the information processing
capability in application, which is the foundation of the ANN. Two critical
issues are involved in the search process of an ANN architecture: the
representation and the search operators.
Figure \ref{fig:evolution} summarizes different these four levels of evolution in ANN's.



\subsection{Search Strategy}
To use the GA method in this work, we need to represent the ANN, devise a
fitness function that determines how good a solution is, and decides the
genetic search operators, including selection, mutation, and crossover.

For the representation of an ANN, encode the $h_i$ node as a eighteen digits
binary string. The initial sixteen digits in the string correspond to the
connections between $i_i$ and $h_i$, with '1' implying there exists a
connection between them, with '0' implying no connection exists. The last two
digits in the string refer to a activation function, such as "01" means a
sigmoid fuction.  Tab.\ref{tab:binary-rep} are examples of the binary
representation of ANNs whose architecture is as shown in
Fig.\ref{fig:anns}. 

For the objective function, treat the multiplicative inverse of the mean
squared error, which is the difference between the target and actual output
averaged overall examples, as the fitness function.

The crossover between individuals results in exploiting the area between the
given two parent solutions. In the present study, we search the local area by
combining the genes of half number of nodes from both parents.
Fig.\ref{fig:anns} illustrates the crossover operator: Fig.\ref{fig:anns}
(c) is the child of Fig.\ref{fig:anns} (a) and Fig.\ref{fig:anns} (a), the
connection relationship of hidden nodes with inputs are from both parents,  and
the corresponding activation functions are also from both parents. In the
binary representation Tab.\ref{tab:binary-rep}, we can see that the first two
rows of the child is the same as the first two rows of parent $P_1$, and the
last six rows of the child is the same as the first six rows of parent $P_2$.

Mutations introduces small changes in children produced by crossover.

First,  randomly initialize the population, partial training every ANN, 





\subsection{Performance estimation strategy}
The simplest approach to this problem is to perform a standard training and
validation of the architecture on dataset, however, this method is inefficient
and computational intensive. Therefore, much recent
research\cite{baker2017accelerating} foces on developing methods that reduce
the cost of performance estimation. In this work, during the GA process we adopt the following
straigtheforward and efficient method to estimate the performance of an ANN:
first, train a neural network one hundred times on the train dataset;
second, do the validation test; estimate the neural network's performance
according to its fitness of objective function on the test dataset.  After get
the best ANN, then do a standard training.



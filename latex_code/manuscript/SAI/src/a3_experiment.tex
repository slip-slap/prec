\section{Experiment}
In the previous section we present the details of our strategies for designing an ANN. In
this section, we describe the details of praparation of training set, and validation set.
\subsection{Dataset Preparation}
For compoiste material, it is impossible to obtain massive training data from
practical scenerio.  Therefore, we use classical lamination theory and failure
theory to prepare the training dataset, which follows a two-step procedure to
calculate the strength ratio: first, evaluate the stress and strain according
to classic lamination theory; second, substitue them into the corresponding
equation to get the strength ratio. We repeat this procedure to yield 14000
points uniformly distributed over the domain space.

The range of in-plane loading is from 0 to 120; the range of fiber orientation $\theta$ is from
-90 to 90; ply thickness $t$ is 1.27mm, number of plies range $N$ is from 4 to 120;
Three different material is used in this experiment, as shown in table \ref{tab:mat}.
Figure \ref{tab:traing-data} shows part of the training data.

In order to speeds up the learning and accerlate convergence, the input
atttributes of the dataset are rescaled to between 0 and 1.0 by a linear function.



\begin{table*}[!t]	
\centering
\caption{Examples of the training data}
\label{tab:traing-data}
\begin{adjustbox}{width=1\textwidth}
	\begin{tabular}{cccc|cc}
		\toprule
		\multicolumn{4}{c}{\textbf{Input}} &  \multicolumn{2}{c}{\textbf{Output}} \\
		\midrule
		Load  &  \makecell{Laminate \\ Structure }  & \makecell{Material \\ Property} & \makecell{Failure \\  Property}  & MS & Tsai-Wu \\
		\midrule

		-70,-10,-40,  & 90,-90,4,1.27, & 38.6,8.27,0.26,4.14,  & 1062.0,610.0,31,118,72,  & 0.0102, & 0.0086 \\
		-10,10,0,     & -86,86,80,1.27,& 181.0,10.3,0.28,7.17, & 1500.0,1500.0,40,246,68, & 0.4026, & 2.5120 \\
		-70,-50,80,   & -38,38,4,1.27, & 116.6,7.67,0.27,4.173,& 2062.0,1701.0,70,240,105,& 0.0080, & 0.0325 \\
		-70,80,-40,   & 90,-90,48,1.27,& 38.6,8.27,0.26,4.14,  & 1062.0,610.0,31,118,72,  & 0.0218, & 0.1028 \\
		-20,-30,0,    & -86,86,60,1.27,& 181.0,10.3,0.28,7.17, & 1500.0,1500.0,40,246,68, & 0.6481, & 0.9512 \\
		0,-40,0,      & 74,-74,168,1.27,& 181.0,10.3,0.28,7.17,& 1500.0,1500.0,40,246,68, & 1.3110, & 3.9619 \\
		\bottomrule
		\end{tabular}
\end{adjustbox}
\end{table*}


The ANN training procedure is carried out by optimising the multinomial
logistic regression objective using mini-batch gradient descent\cite{lecun1989backpropagation} with momentum. the batch size is set to 1000,
momentum to 0.9. the learning rate is set to $10^{-2}$.


\section{Result and Discussion}
We have conducted experiment by the use of the generated data set. This data
set is randomly partitoned into a training set and a test.

Figure \ref{fig:train-process} shows five ANNs with different topologies, quite
different results have been observed when different architectures are adopted.
It is clear that architecture whose mean of average difference is less than the
rest.


\begin{figure}
	\centering
	\def\svgwidth{\columnwidth}
	\import{fig/}{pre_train.pdf_tex}
	\label{fig:train-process}
\end{figure}

Table \ref{tab:simu} shows part of the valudation.
\begin{table}	
	\centering
	\caption{Comparsion between practical and simulation}
	\label{tab:simu}
		\resizebox{\textwidth}{!}{
	\begin{tabular}{cccc|cc|cc}
		\toprule
		\multicolumn{4}{c}{\textbf{Input}} &  \multicolumn{4}{c}{\textbf{Output}} \\
		\midrule
		Load  &  \makecell{Laminate \\ Structure }  & \makecell{Material \\ Property} & \makecell{Failure \\  Property}  &
		\multicolumn{2}{c}{ \makecell {CLT \\MS  Tsai-Wu}} & \multicolumn{2}{c}{ \makecell {ANN \\MS  Tsai-Wu}}\\
		\midrule
		-10,40,20  &  26,-26,168,1.27 & 116.6,7.67,0.27,4.17 & 2062.0,1701.0,70,240,105 & 0.342 & 0.476 & 0.351 & 0.492 \\
		20,-70,-30 &  10,-10,196,1.27 & 181.0,10.3,0.28,7.17 & 1500.0,1500.0,40,246,68  & 0.653 & 0.489 & 0.612 & 0.445 \\ 
		60,-20,0   &  82 -82,128,1.27 & 181.0,10.3,0.28,7.17 & 1500.0,1500.0,40,246,68  & 1.663 & 0.112 & 1.673 & 0.189 \\
		\bottomrule
	\end{tabular}
	}
\end{table}



\begin{figure}
	\centering
	\def\svgwidth{\columnwidth}
	\import{fig/}{post_train.pdf_tex}
	\label{fig:final_train}
\end{figure}


Comparing the strength ratio outputs based on CLT and ANN from
Tab.\ref{tab:simu}, we can see that the calculation of strength ratio can be
achieved using a two-layer neural network, without the intensive computation of
matrix multiplication.





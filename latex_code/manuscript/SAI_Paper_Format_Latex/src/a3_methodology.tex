\section{Genetic algorithm model}

\input{fig/chapter3_ngam.tex}
\input{fig/chapter3_ogam.tex}
The genetic algorithm starts with multiple individuals with limited chromosome lengths, in
which maybe none of these individuals fulfill the constraints. The GA is
assumed to derive appropriate offspring based on the initial population as the
GA continues. The traditional way to handle the constrained search of the GA is
either to introduce repair strategies or to use a penalty function. Fig.
\ref{fig:old_ga_model} shows the classic flow chart of a GA framework, which
includes selection, crossover, and mutation operators. However, GA is originally
proposed to solve unconstrained problems; therefore, we suggest a new approach 
to address the constrained GA search problem in an unconstrained way. 

Because of the existence of constraints, the population not only can be sorted
by the fitness (obtained by the objective function) but also sorted by
the constraint value obtained by the constraint function (assuming a constraint
function exists), so the parents of the next generation can be chosen by the
following three approaches. First, the population is sorted by fitness in
ascending order, and individuals with smaller fitness are selected. These
selected individuals form a group named as a proper group. Second, remove
individual which satisfies constraints, and sort population by the difference
between the individual's constraint value and the threshold of the constraint
in descending order, and individuals with bigger differences are chosen to be
the parents of the next generation. The group which forms are called potential
group, and an individual from this group is referred to as a potential
individual.  Third, the population is sorted by fitness from low to high after
removing individuals which fails to fit the constraints, select individuals
with bigger fitness, and these individuals form the proper group.  So the final
parents' pool is consists of three groups, active group, potential group, and
proper group.  The number of active individuals, potential individuals, and
proper individuals are called, respectively, active number, potential numbers,
and proper number. 

Each group in the parents' population has its role in the searching
process. The problem within traditional GA is premature and has weak local
search ability, therefore, traditional GAs are more likely to get stuck in a
local optimum. To prevent the GA from experiencing early convergence and to
improve the local search performance, the active group is proposed to overcome
this problem. As its name suggests, this group would always live in the
population.  Because both active individual's fitness and constraint value are small,
each individual can be treated as an independent gene clip. So their existence
enriches the gene clip variety of the mating pool. The offspring of two active
individuals are more likely to be an active individual, which can maintain the
active group.

For an individual in the potential group, it doesn't satisfy the constraint,
however, it's supposed to evolve into a proper individual after multiple
generations by modifying its chromosome structure or length. The crossover
operation could happen between a potential individual with an active
individual, or a potential individual, or a proper individual. The child of an
active individual and a potential individual is more likely to be a potential individual,
and this active individual could inject a new gene clip into this potential
individual, therefore providing a new evolution direction. 


A proper individual means a feasible solution, which fulfills all the constraints. 
However, there are still some drawbacks within it, for example, its fitness is low.
The crossover operation could happen between a proper individual and any other individuals.


The mutation operator for an active group is different from the potential group
and proper group because their roles in the searching process are different:
the target of the potential group and proper group is to obtain a feasible
solution; however, the role of the active group is to maintain the variety of gene
clips in the mating pool. 

Fig.\ref{fig:model} shows the flow chart of the proposed GA. First, the
population is divided into three groups, active group, potential group, and
proper group by the above-mentioned method. Second, select an appropriate number of
individuals from each group as parents, and the various selection scheme can be
taken for each group. 


The searching process can be divided into two phases according to whether
proper individuals are generated or not. During the initial stage, no individual in
the population is appropriate, which means the number of individuals in the
proper group is zero. Both active group and potential group are full. After a
couple of generations, some proper individuals could be produced. Then, the GA
comes to its second phase, the number of proper individuals begins to increase,
finally, the number in the proper group reaches its upper bound. 


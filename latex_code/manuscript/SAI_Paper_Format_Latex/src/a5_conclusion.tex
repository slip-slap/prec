\section{conclusions}
In this paper, we reviewed the use of the proposed ga framework, classical
lamination theory, and tsai-wu failure theory for the layup design for
cross-ply laminate. Because GA is primarily used to solve an unconstrained
problem, and it is not suitable for a constrained problem. In the present
study, we deal with this constrained problem by mixing strategies of selection
methods instead of adding punishment terms into the objective function. So the
constraint problem can be solved in an unconstrained way.

This variant of the GA provides a new approach to address the constrained
search for optimization of laminated composite, and this method can be easy
to apply in other domains. At the same time, the proposed GA model is more
complicated than the traditional GA model, which involves more parameters. To
advance its performance, the fine-tuning of those parameters need more effort. 

\section{conclusions}
In this paper, we reviewed the use of the proposed genetic algorithm framework, classical lamination theory, and Tsai-Wu failure theory for the layup design for cross-ply laminate under different loading cases. The principal applications of this genetic algorithm are in the design of composite laminate material with imposed constraints.

The main contribution of the present work is the genetic algorithm framework for constrained problems since the traditional genetic algorithm is primarily concerned with solving the unconstrained problem, which is not suitable for a constrained case. We deal with these constrained problems by the use of mixing strategies of selection methods instead of adding any punishment terms into the objective
function. The performance of this algorithm is heavily influenced by
the coefficient of the length mutation. Both for the glass-epoxy and graphite-epoxy material cases, if the coefficient takes a relatively low value, this algorithm can obtain better results than when the coefficient value is high. However, The algorithm converged more quickly with a high coefficient value of length mutation. 

This variant of the genetic algorithm provides a new approach to address the constrained search, and this method can be easy to apply in other domains. A drawback of the proposed genetic algorithm is more parameters involving in this GA, which makes fine-tuning more difficult.

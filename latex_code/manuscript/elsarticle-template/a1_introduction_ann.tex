\section{Artificial Neural Network}
Artificial neural networks(ANN) which heavily inspired by biology and psychology
have been widely used to solve various practical engineering problems in such
areas as pattern recognition, nonlinear regression, data mining, clustering and
prediction. Methods based on complicated mathematical models is of intensive
computation, approximation function evaluation techniques can be employed to
accelerate the calculation process and reduce the computation cost. In order to
solve practical engineering problems in composite material application, classic
laminational theory(CLT) has been proposed which involves many matrix
multiplication and integration calculation. ANN which has been proved is a
reliable tool instead of complicate mathematical model.  The design of neural
network consists of three basic parts: neural network architecture, learning
rules, and training techniques.

The weight training in ANN is to minimize the error function, such as the most
widely used mean square error which calculate the difference  between the
desired and the prediction output values averaged over all examples.
Backpropation algorithm has been successful applied to in many areas, and it's
based on gradient descent. However, this class of algorithms are  plagued by the
possible existance of local minima or "flat spots" and "the curse of
dimensionality". One method to overcome this problem is to adopt EANN's 


\begin{table}
\centering
\caption{Different Activation Functions}

\begin{tabular}{lllc}
		\toprule
		Type & Description  & Formula & Range  \\
		\midrule
		Linear & The output is proportional to the input & $f(x)=cx$ &
		$(-\infty, +\infty)$ \\ 
		Sigmoid & A family of S-shaped functions& $f(x)=\frac{1}{1+e^{-cx}}$ &
		$(0, 1)$ \\ 
		Gaussian & A coninuous bell-shaped curve & $f(x)=e^{-x^2}$ & $(0,1)$ \\ 
		\bottomrule
	\end{tabular}
\end{table}


\begin{figure}
\centering
\begin{tikzpicture}
	\begin{scope}
		%\draw[style=help lines] (-3,-3) grid (3,3);
		\draw (0,0) rectangle (2,3);
		\draw[->] (1.3,1.2) -- (2.6,1.2);
		\draw[->] (1.3,1.2) -- (1.3,3.4);
		\node at (2.2,1) {$X_T$};
		\node at (1.5, 3.2) {$Y_T$};
		\node at (-0.2, 0.9) {$X_C$};
		\node at (1.8, -0.2) {$Y_C$};
	\end{scope}
	\begin{scope}[xshift=6cm,yshift=1.15cm]
		%\draw[style=help lines] (-3,-3) grid (3,3);
		\draw[rotate=30] (0,0) ellipse (2cm and 1cm);
		\draw[->] (0.2,0) -- (0.2,2.2);
		\draw[->] (0.2,0) -- (1.9,0);
		\node at (1.6,-0.2) {$X_T$};
		\node at (0.3, 1.3) {$Y_T$};
		\node at (-1.6, 0) {$X_C$};
		\node at (-0.5, -1.5) {$Y_C$};
	\end{scope}
\end{tikzpicture}
\caption{Schematic failure surfaces for maximum stress and quadratic failure
criteria}
\end{figure}



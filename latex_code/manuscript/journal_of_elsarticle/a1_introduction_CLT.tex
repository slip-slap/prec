\section{Introduction}
Fiber-reinforced composite materials have been widely used in many industries
because they offer improved mechanical stiffness, strength, and low specific
gravity of fibers  over conventional materials. The use of composite material
materials in structural application is range from electronic packaging, sports
equipment, homebuilding, medical prosthetic devices, to high performance
military structures. The stacking sequence, ply thickness, and fiber
orientation of composite laminates give the designer additional ’degree of
freedom’ to tailor the design with respect to strength or stiffness. Classic
lamination theory(CLT) is taken to predict the behavior of a laminate from a
knowledge of the composite laminate properties of the individual layers and the
laminate geometry.

Evolutionary artificial neural networks(EANN's) is a special class of artifical
neural networks(ANN's) in which evolutionary algorithms(ES's) are introduced to
learn the optimal ANN. EA's can be used in the ANN at three different levels:
connection weights, architectures, input features, and learning rules. It is
shown, the combinations of ANN's and EA's can significantly improve the
performance of intelligent systems that rely's on ANN's or EA's alone.



The rest of the paper is organized as follows. Section 2 explains the classical
laminate theory and the failure criteria taken in the present study. Section 3
explains the design of artifical neural network for mathmatical model
approximation.  Section 4 reviews the use of genetic algorithm in the design of
neural network architecture and the parameters optimization during the training
process of neural network.  design Section 4 describes the result of the
numerical experiments in different cases, and in the Conclusion section we
dicuss the results.
\section{Classic Lamination Theory}
Classical lamination theory is based upon three simplifying engineering
assumptions: (1) Each layer's thickness is very small and consist of
homogeneous, orthotropic material, and these layers are prefectly bonded
together; (2) The entire laminated composite is supposed to be under plane
stress; (3) Normal cross sections of the entire laminate is normal to the
deflected middle surface, and do not change in thickness.
\subsection{Stress and Strain in a Lamina}
For a single lamina has a small thickness under plane stress, and it's upper and lower surfaces of the lamina are
free from external loads. According to the Hooke's Law, the three-dimensional stress-strain equations can be reduced to
two-dimensional stress-strain equations. The stress-strain relation in local axis 1-2 is:
\begin{equation}
    \begin{bmatrix}
        \sigma _1\\
        \sigma _2\\
        \tau_{12}
    \end{bmatrix}
    =
    \begin{bmatrix}
        Q_{11} & Q_{12} & 0\\
        Q_{12} & Q_{22} & 0\\
        0      &  0     & Q_{66}
    \end{bmatrix}
    \begin{bmatrix}
        \varepsilon_1\\
        \varepsilon_2\\\gamma_{12}
    \end{bmatrix}
\end{equation}
where $Q_{ij} $are the stiffnesses of the lamina that are related

to engineering elastic constants given by
\begin{equation}
    \begin{split}
    &Q_{11}=\frac{E_1}{1-v_{12}v_{21}}\\
    &Q_{22}=\frac{E_2}{1-v_{12}v_{21}}\\
    &Q_{66}=G_{12}\\
    &Q_{12}=\frac{v_{21}E_2}{1-v_{12}v_{21}}\\
    \end{split}
\end{equation}

where $E_1, E_2, v_{12}, G_{12} $ are four independent engineering elastic constants, which are defined as follows: $E_1 $ is the longitudinal Young's modulus, $E_2 $ is the transverse Young's modulus, $v_{12} $ is the major Poisson's ratio, and $G_{12} $ is the in-plane shear modulus.

Stress strain relation in the global x-y axis:
\begin{equation}
	\label{equ:stress-strain}
	\left[\begin{array}{l}
			\sigma _{x} \\ \sigma _{y} \\
			\tau_{xy}\end{array}\right]=\left[\begin{array}{lll}\bar{Q}_{11} &
			\bar{Q}_{12} & \bar{Q}_{16}\\ 
			\bar{Q}_{12} & \bar{Q}_{22} & \bar{Q}_{26} \\
								\bar{Q}_{16} & \bar{Q}_{26}
			 &\bar{Q}_{66}\end{array}\right]\left[\begin{array}{l}\varepsilon_{x}
	 \\ \varepsilon_{y}\\ \gamma_{x y}\end{array}\right] \end{equation}
where
\begin{equation}
	\begin{array}{l}
		\resizebox{.35\textwidth}{!}{$\bar{Q}_{11}=Q_{11} cos^{4}\theta+Q_{22} sin^{4}\theta+2\left(Q_{12}+2
		Q_{66}\right) sin^{2}\theta cos^{2}\theta$} \\

		\resizebox{.35\textwidth}{!}{$\bar{Q}_{12}=\left(Q_{11}+Q_{22}-4 Q_{66}\right) sin^{2}\theta
		cos^{2}\theta+Q_{12}\left(cos^{4}\theta+sin^{2}\theta \right)$} \\

		\resizebox{.35\textwidth}{!}{$\bar{Q}_{22}=Q_{11} sin^{4}\theta+Q_{22} cos^{4}\theta+2\left(Q_{12}+2
		Q_{66}\right) sin^{2}\theta cos^{2}\theta$} \\

		\resizebox{.4\textwidth}{!}{$\bar{Q}_{16}=\left(Q_{11}-Q_{12}-2
		Q_{66}\right) cos^{3}\theta sin\theta-\left(Q_{22}-Q_{12}-2Q_{66}\right)
	sin^{3}\theta cos\theta$} \\ 
		\resizebox{.4\textwidth}{!}{$\bar{Q}_{26}=\left(Q_{11}-Q_{12}-2
		Q_{66}\right) cos\theta sin^{3}\theta-\left(Q_{22}-Q_{12}-2
Q_{66}\right)cos^{3}\theta sin\theta$}
		 \\ 
	\resizebox{.4\textwidth}{!}	{$\bar{Q}_{66}=\left(Q_{11}+Q_{22}-2 Q_{12}-2 Q_{66}\right)
	sin\theta^{2}cos\theta^{2}+Q_{66}\left(sin\theta^{4}+cos\theta^{4}\right)$}\\
	\end{array}
\end{equation}



The local and global stresses in an angle lamina are related

to each other through the angle of the lamina $\theta $
\begin{equation}\left[\begin{array}{l}\sigma _{1} \\ \sigma _{2} \\ \tau_{12}\end{array}\right]=[T]\left[\begin{array}{l}\sigma _{x} \\ \sigma _{y} \\\tau_{xy}\end{array}\right]
\end{equation}

where
\begin{equation}
	[T]=\left[\begin{array}{ccc}cos^{2}\theta & sin^{2}\theta & 2
		sin\theta cos\theta \\ 
sin^{2}\theta & cos^{2}\theta & -2 sin\theta cos\theta \\
-sin\theta cos\theta
			  & sin\theta cos\theta  &cos^{2}\theta -sin^{2}\theta
\end{array}\right] 
\end{equation}



\subsection{Stress and Strain in a Laminate}
For forces and moment resultants acting on laminates, such as in plate and shell
structures, the relationship between applied forces and moment and displacement
can be given by

\begin{equation} \label{eq:force_and_moments}
	\begin{array}{l}
		\begin{aligned}
	\begin{bmatrix}
		N_x \\
		N_y \\
		N_{xy}
	\end{bmatrix}
	&=
	\begin{bmatrix}
		A_{11} & A_{12} & A_{16} \\
		A_{12} & A_{22} & A_{26} \\
		A_{16} & A_{26} & A_{66} 
	\end{bmatrix}
    \begin{bmatrix}
		\varepsilon_x^0 \\
        \varepsilon_y^0 \\
		\gamma_{xy}^0
    \end{bmatrix}   \\
	&+               
	\begin{bmatrix}
		B_{11} & B_{12} & B_{16} \\
		B_{11} & B_{12} & B_{16} \\
		B_{16} & B_{26} & B_{66} 
	\end{bmatrix}
	\begin{bmatrix}
		k_x \\
		k_y \\
		k_{xy} 
	\end{bmatrix}  \\
\end{aligned} \\ \\
\begin{aligned}
	\begin{bmatrix}
		M_x \\
		M_y \\
		M_{xy}
	\end{bmatrix}
	&=
	\begin{bmatrix}
		B_{11} & B_{12} & B_{16} \\
		B_{12} & B_{22} & B_{26} \\
		B_{16} & B_{26} & B_{66} 
	\end{bmatrix}
    \begin{bmatrix}
		\varepsilon_x^0 \\
        \varepsilon_y^0 \\
		\gamma_{xy}^0
    \end{bmatrix} \\ 
	&+  
	\begin{bmatrix}
		D_{11} & D_{12} & D_{16} \\
		D_{11} & D_{12} & D_{16} \\
		D_{16} & D_{26} & D_{66} 
	\end{bmatrix}
	\begin{bmatrix}
		k_x \\
		k_y \\
		k_{xy} 
	\end{bmatrix}
\end{aligned}
	\end{array}
\end{equation}


$N_x,N_y $  - normal force per unit length

$N_{xy} $  - shear force per unit length

$M_x, M_y $ - bending moment per unit length

$M_{xy} $  - twisting moments per unit length

$\varepsilon^{0}, k $- mid plane strains and curvature of a laminate in x-y coordinates

The mid plane strain and curvature is given by
\begin{equation}
    \begin{split}
    &A_{ij}=\sum_{k=1}^{n}(\overline{Q_{ij}})_k(h_k-h_{k-1})  i=1,2,6, j=1,2,6\\
    &B_{ij}=\frac{1}{2}\sum_{k=1}^{n}(\overline{Q_{ij}})_k(h_k^2 - h_{k-1}^2)  i=1,2,6, j=1,2,6\\
    &D_{ij}=\frac{1}{3}\sum_{k=1}^{n}(\overline{Q_{ij}})_k(h_k^3 - h_{k-1}^3) i=1,2,6, j=1,2,6\\
    \end{split}
\end{equation}

The [A], [B], and [D] matrices are called the extensional, coupling, and bending stiffness matrices,
respectively. The extensional stiffness matrix $[A]$ relates the resultant in-plane forces to the
in-plain strains, and the bending stiffness matrix $[D]$ couples the resultant bending moments to
the plane curvatures.  The coupling stiffness matrix $[B]$ relates the force and moment terms to the
midplain strains and midplane curvatures.

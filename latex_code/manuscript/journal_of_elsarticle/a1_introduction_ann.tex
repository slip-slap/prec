\section{Artificial Neural Network}
Artificial neural networks(ANN) which heavily inspired by biology and psychology
have been widely used to solve various practical engineering problems in such
areas as pattern recognition, nonlinear regression, data mining, clustering and
prediction. Methods based on complicated mathematical models is of intensive
computation, approximation function evaluation techniques can be employed to
accelerate the calculation process and reduce the computation cost. In order to
solve practical engineering problems in composite material application, classic
laminational theory(CLT) has been proposed which involves many matrix
multiplication and integration calculation. ANN which has been proved is a
reliable tool instead of complicate mathematical model.  The design of neural
network consists of three basic parts: neural network architecture, learning
rules, and training techniques.

The weight training in ANN is to minimize the error function, such as the most
widely used mean square error which calculate the difference  between the
desired and the prediction output values averaged over all examples.
Backpropation algorithm has been successful applied to in many areas, and it's
based on gradient descent. However, this class of algorithms are  plagued by
the possible existance of local minima or "flat spots" and "the curse of
dimensionality". One method to overcome this problem is to adopt EANN's.

\subsection{General neural network}
\begin{figure*}
\begin{tikzpicture}
[ plain/.style={ draw=none, fill=none, }, remember picture, net/.style={ matrix of nodes, nodes={ draw, circle,
    inner sep=7.5pt
    },
  nodes in empty cells,
  column sep=-10.5pt,
  row sep=0.8cm
  }
]
%\draw[help lines] (-3cm,-6cm) grid (6cm,3cm);
\matrix[net] (mat)
{
              & |[plain]| &           & |[plain]|  &           & |[plain]| &           &  |[plain]|      &               \\
    |[plain]| &           & |[plain]| &            & |[plain]| &           & |[plain]| &                 & |[plain]|     \\ 
    |[plain]| & |[plain]| &           & |[plain]|  &           & |[plain]| & 	  	   &  |[plain]|      & |[plain]|     \\ 
  };

  \node at ($(mat-1-1.west)+(-16pt,0)$) {Input: };
  \node at ($(mat-2-2.west)+(-24pt,0)$) {Hidden:};
  \node at ($(mat-3-2.west)+(-24pt,0)$) {Output:};
  \node at (mat-1-1.base) {$i_1$};
  \node at (mat-1-3.base) {$i_2$};
  \node at (mat-1-5.base) {...};
  \node at (mat-1-7.base) {$i_{n-1}$};
  \node at (mat-1-9.base) {$i_{n}$};
  \node at (mat-2-2.base) {$h_1$};
  \node at (mat-2-4.base) {$h_2$};
  \node at (mat-2-6.base) {$...$};
  \node at (mat-2-8.base) {$h_{m}$};
  \node at (mat-3-5.base) {$...$};

 \foreach \a in {1,3}{
    \foreach \b in {2,6}{
        \draw[->] (mat-1-\a.south) -- (mat-2-\b.north);
     }
  }
 \foreach \a in {3,7,9}{
    \foreach \b in {4,8}{
        \draw[->] (mat-1-\a.south) -- (mat-2-\b.north);
     }
  }

 \foreach \c in {2,4,6,8}{
    \foreach \d in {3,5,7}{
 		\draw[->] (mat-2-\c.south) -- (mat-3-\d.north);
	}
 }
\end{tikzpicture}
\caption{Neural Network Model}
\end{figure*}



\begin{figure}
\centering
\begin{tikzpicture}
    %\draw[help lines] (-3cm,-6cm) grid (6cm, 6cm);
    \tikzstyle{block} = [rectangle, text centered, draw=black,
    minimum width=1.1cm, minimum height=0.4cm]
    % first level
    \node (evaluation-parent) [block, minimum width=2.4cm, minimum
        height=1.8cm,draw=white] {};
    \node (evaluation) [block] at ($(evaluation-parent.north)$) {evaluation};
    \node (reproduction) [block] at ($(evaluation-parent.south)$) {reproduction};
    \node (tasks) [block, minimum width=1.1cm, minimum height=0.4cm] {tasks};

    \draw[->] ($(evaluation.south)+(0.3cm,0cm)$) --
        ($(tasks.north)+(0.3cm,0cm)$) node[auto=left, pos=0.5] {\small weights}; 
    \draw[<-] ($(evaluation.south)+(-0.3cm,0cm)$) --
        ($(tasks.north)+(-0.3cm,0cm)$) node[auto=right, pos=0.5] {\small fitness}; 

    % get intersection
    \draw[white] (evaluation.west) coordinate (A) -- ++(-1.5cm,0) coordinate (B);
    \draw[white] (reproduction.west) -- ++(-0.3cm,0) coordinate (C) -- ++(0,4cm) coordinate
        (D);
    \draw[black] (reproduction.west) -- ++(-0.3cm,0) -- (intersection cs:
        first line={(A)--(B)}, second line={(C)--(D)}) coordinate (E);
    \draw[->] (E) -- (evaluation.west);

    \draw[white] (evaluation.east) coordinate (E) -- ++(2cm,0) coordinate (F);
    \draw[white] (reproduction.east) -- ++(0.3cm,0) coordinate (G) -- ++(0,4cm) coordinate
        (H);
    \draw[<-] (reproduction.east) -- ++(0.3cm,0) -- (intersection cs:
        first line={(E)--(F)}, second line={(G)--(H)}) coordinate (I);
    \draw (I) -- (evaluation.east);

    % second level
    \node (level2) [block,draw=black, minimum width=3.5cm, minimum height=3.0cm] at
        (0cm,0.2cm) {};
    \node [align=left] at ($(level2.north)+(0,-0.2cm)$) {\tiny THE EVOLUTION
        OF};
    \node [align=left] at ($(level2.north)+(0,-0.45cm)$) {\tiny CONNECTION
            WEIGHTS 
        };
    % third level
    \node (level3-assister) [block, draw=white, minimum width=5cm, minimum
		height=4.6cm] at
        (0, 0.3cm)  {};
    \node (evaluation) [block] at ($(level3-assister.north)$) {\small evaluation of
        learning rules};
    \node (reproduction) [block] at ($(level3-assister.south)$) {\small reproduction of
        learning rules};

    \draw[->] ($(evaluation.south)+(0.3cm,0cm)$) --
        ($(level2.north)+(0.3cm,0cm)$) node[auto=left, pos=0.5] {\small learning
        rule}; 
    \draw[<-] ($(evaluation.south)+(-0.3cm,0cm)$) --
        ($(level2.north)+(-0.3cm,0cm)$) node[auto=right, pos=0.5] {\small fitness}; 

    \draw[white] (evaluation.west) coordinate (A) -- ++(-1.3cm,0) coordinate (B);
    \draw[white] (reproduction.west) -- ++(-0.3cm,0) coordinate (C) -- ++(0,4cm) coordinate
        (D);
    \draw[black] (reproduction.west) -- ++(-0.3cm,0) -- (intersection cs:
        first line={(A)--(B)}, second line={(C)--(D)}) coordinate (E);
    \draw[->] (E) -- (evaluation.west);

    \draw[white] (evaluation.east) coordinate (E) -- ++(2cm,0) coordinate (F);
    \draw[white] (reproduction.east) -- ++(0.3cm,0) coordinate (G) -- ++(0,4cm) coordinate
        (H);
    \draw[<-] (reproduction.east) -- ++(0.3cm,0) -- (intersection cs:
        first line={(E)--(F)}, second line={(G)--(H)}) coordinate (I);
    \draw (I) -- (evaluation.east);
   % fourth level
    \node (level4) [block, draw=black, minimum width=5.5cm, minimum
        height=6.0cm] at
        (0, 0.4cm)  {};
    \node [align=left] at ($(level4.north)+(0,-0.25cm)$) {\tiny THE EVOLUTION
        OF LEARNING RULES};
    % level five
    \node (level5-assister) [block, draw=white, minimum width=6.4cm, minimum
        height=7.0cm] at
        (0, 0.4cm)  {};
    \node (evaluation) [block] at ($(level5-assister.north)$) {\small evaluation of
        architecture};
    \node (reproduction) [block] at ($(level5-assister.south)$) {\small reproduction of
        learning architecture};

    \draw[white] (evaluation.west) coordinate (A) -- ++(-1.5cm,0) coordinate (B);
    \draw[white] (reproduction.west) -- ++(-0.6cm,0) coordinate (C) -- ++(0,4cm) coordinate
        (D);
    \draw[black] (reproduction.west) -- ++(-0.6cm,0) -- (intersection cs:
        first line={(A)--(B)}, second line={(C)--(D)}) coordinate (E);
    \draw[->] (E) -- (evaluation.west);

    \draw[white] (evaluation.east) coordinate (E) -- ++(2cm,0) coordinate (F);
    \draw[white] (reproduction.east) -- ++(0.6cm,0) coordinate (G) -- ++(0,4cm) coordinate
        (H);
    \draw[<-] (reproduction.east) -- ++(0.6cm,0) -- (intersection cs:
        first line={(E)--(F)}, second line={(G)--(H)}) coordinate (I);
    \draw (I) -- (evaluation.east);
    % level 6
    \node (level6) [block, minimum width=7.0cm, minimum
        height=8.1cm] at
        (0, 0.5cm)  {};
    \node [align=left] at ($(level6.north)+(0,-0.2cm)$) {\tiny THE EVOLUTION OF
            ARCHITECTURE
        };
\end{tikzpicture}
\caption{Genetic algorithm and artificial neural network}
\end{figure}


In this paper, the feedforward nueural network models are adopted in our
system, because feedforward NNs are straigntforward and simply to code.  For
funnction approximation, Cybenko demonstrated that a two-lay multilayer
perceptron(MLP) is capbable of forming an arbitrarily close approximation to
any continusous nonliner mapping\cite{cybenko1989approximation}. Therefore, the


\subsection{Transfer function}
The transfer fucntion has been shown to be one of the critical part of the
architecture. Liu \cite{liu1996evolutionary} et al. have claimed that ANNs with
different active functions play a important role in the arthitecture's performance.

\begin{table}
\centering
\caption{Different Activation Functions}
\begin{adjustbox}{width=1\textwidth}
\label{tab:transfer_function}
	\begin{tabular}{lllc}
			\toprule
			Type & Description  & Formula & Range  \\
			\midrule
			Linear   & The output is proportional to the input & $f(x)=cx$                  &  $(-\infty, +\infty)$ \\
			Sigmoid  & A family of S-shaped functions          & $f(x)=\frac{1}{1+e^{-cx}}$ & $(0, 1)$ \\
			tanh     & A family of Hyperbloic functions        & $f(x)=\frac{e^x -e^{-x}}{e^x+e^{-x}}$ & $(0, 1)$ \\
			Gaussian & A coninuous bell-shaped curve           & $f(x)=e^{-x^2}$            & $(0,1)$ \\ 
			ReLU     & A piece-wise function                   & $f(x)= max{0,x}$           & $(0, +\infty)$ \\
			Softplus & A family of S-shaped functions          & $f(x) = ln(1+e^x)$         & $(0, +\infty)$ \\
			\bottomrule
	\end{tabular}
\end{adjustbox}
\end{table}











\section{Artificial Neural Network}
Artificial neural networks(ANN) which heavily inspired by biology and psychology
have been widely used to solve various practical engineering problems in such
areas as pattern recognition, nonlinear regression, data mining, clustering and
prediction. Methods based on complicated mathematical models is of intensive
computation, approximation function evaluation techniques can be employed to
accelerate the calculation process and reduce the computation cost. In order to
solve practical engineering problems in composite material application, classic
laminational theory(CLT) has been proposed which involves many matrix
multiplication and integration calculation. ANN which has been proved is a
reliable tool instead of complicate mathematical model.  

\subsection{General neural network}
\begin{figure}
	\centering
	\begin{tikzpicture}
		[ plain/.style={ draw=none, fill=none, }, remember picture, net/.style={ matrix of nodes, nodes={ draw, circle,
			inner sep=7.5pt
			},
		  nodes in empty cells,
		  column sep=-10.5pt,
		  row sep=0.8cm
		  }
		]
	%\draw[help lines] (-3cm,-6cm) grid (6cm,3cm);
	\matrix[net] (mat)
		{
				  & |[plain]| &           & |[plain]|  &           & |[plain]| &           &  |[plain]|      &               \\
		|[plain]| &           & |[plain]| &            & |[plain]| &           & |[plain]| &                 & |[plain]|     \\ 
		|[plain]| & |[plain]| &           & |[plain]|  &           & |[plain]| & 	  	   &  |[plain]|      & |[plain]|     \\ 
	  };

	  \node at ($(mat-1-1.west)+(-16pt,0)$) {Input: };
	  \node at ($(mat-2-2.west)+(-24pt,0)$) {Hidden:};
	  \node at ($(mat-3-2.west)+(-24pt,0)$) {Output:};
	  \node at (mat-1-1.base) {$i_1$};
	  \node at (mat-1-3.base) {$i_2$};
	  \node at (mat-1-5.base) {...};
	  \node at (mat-1-7.base) {$i_{n-1}$};
	  \node at (mat-1-9.base) {$i_{n}$};
	  \node at (mat-2-2.base) {$h_1$};
	  \node at (mat-2-4.base) {$h_2$};
	  \node at (mat-2-6.base) {$...$};
	  \node at (mat-2-8.base) {$h_{m}$};
	  \node at (mat-3-5.base) {$...$};

		 \foreach \a in {1,3}{
			\foreach \b in {2,6}{
				\draw[->] (mat-1-\a.south) -- (mat-2-\b.north);
			 }
		  }
		 \foreach \a in {3,7,9}{
			\foreach \b in {4,8}{
				\draw[->] (mat-1-\a.south) -- (mat-2-\b.north);
			 }
		  }

		 \foreach \c in {2,4,6,8}{
			\foreach \d in {3,5,7}{
				\draw[->] (mat-2-\c.south) -- (mat-3-\d.north);
			}
	 }
\end{tikzpicture}
\caption{General Neural Network}
\label{fig:gnn}
\end{figure}





In this paper, the feedforward nueural network models are adopted in our
system, because feedforward NNs are straigntforward and simply to implement.  For
funnction approximation, Cybenko demonstrated that a two-lay multilayer
perceptron(MLP) is capbable of forming an arbitrarily close approximation to
any continusous nonliner mapping\cite{cybenko1989approximation}. Therefore, a
two layers feedforward neural network is proposed in present study, for nodes
in the hidden layer, they are in essence feature extractors and detectors.
Therefore, every nodes in the hidden layer should be partial connected within
the inputs, the unnecessary connections would increased the models complicacy
which would reduce the ANN's performance. For the nodes in the last layer,
every node should be full connected with nodes in previou. The design of neural
network consists of three basic parts: neural network architecture, learning
rules, and training techniques.
s layer.


\subsection{Transfer function}
The transfer fucntion has been shown to be one of the critical part of the
architecture. Liu \cite{liu1996evolutionary} et al. have claimed that ANNs with
different active functions play a important role in the arthitecture's performance.
A generalized transfer function can be written as

\begin{equation}
	y_i = f_i(\sum_{j=1}^n{w_{ij}x_j - \theta})
\end{equation}

where $y_i$ is the output of the node $i$, $x_j$ is the $j$th input to the node,
and $w_{ij}$ is the connection weight between adjacent nodes $i$ and $j$. Most
widely transfer function  $f_i$ is listed in table \ref{tab:transfer_function}

\begin{table*}[!t]
\centering
\caption{Different Activation Functions}
\label{tab:active_function}
\begin{adjustbox}{width=1\textwidth}
\label{tab:transfer_function}
	\begin{tabular}{lllc}
			\toprule
			Type & Description  & Formula & Range  \\
			\midrule
			Linear   & The output is proportional to the input & $f(x)=cx$                  &  $(-\infty, +\infty)$ \\
			Sigmoid  & A family of S-shaped functions          & $f(x)=\frac{1}{1+e^{-cx}}$ & $(0, 1)$ \\
			tanh     & A family of Hyperbloic functions        & $f(x)=\frac{e^x -e^{-x}}{e^x+e^{-x}}$ & $(0, 1)$ \\
			Gaussian & A coninuous bell-shaped curve           & $f(x)=e^{-x^2}$            & $(0,1)$ \\ 
			ReLU     & A piece-wise function                   & $f(x)= max{0,x}$           & $(0, +\infty)$ \\
			Softplus & A family of S-shaped functions          & $f(x) = ln(1+e^x)$         & $(0, +\infty)$ \\
			\bottomrule
	\end{tabular}
\end{adjustbox}
\end{table*}


\subsection{Weights learning}
The weight training in ANN is to minimize the error function, such as the most
widely used mean square error which calculate the difference  between the
desired and the prediction output values averaged over all examples.
Backpropation algorithm has been successful applied to in many areas, and it's
based on gradient descent. However, this class of algorithms are  plagued by
the possible existance of local minima or "flat spots" and "the curse of
dimensionality". One method to overcome this problem is to adopt
EANN's\cite{yao1999evolving} through a systematic search process\cite{elsken2019neural} to build
the topology of ANN.








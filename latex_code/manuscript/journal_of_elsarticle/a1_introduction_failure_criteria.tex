\section{Failure criteria for a lamina}

Failure criteria for composite materials are more difficult to predict due to
structural and material complexity in comparison to isotropic materials. The
failure process of a composite materials can be regarded from microscopic and
macroscopic points of view. Most popular criteria about the failure of an angle
lamina are in terms of macroscopic failure criteria, which are based on the
tensile, compressive and shear strengths. According to the failure surfaces,
these criteria can be classified into two classes: one is called independent
failure mode criteria which includes the maximum stress failure theory, maximum
strain failure theory because their failure envelop are rectangle; another is
called quadratic polynomial which includes Tsai-Wu, Chamis, Hoffman and Hill
criteria because their failure surfaces are of ellipsoidal shape. In the present
study, two most reliable failure criteria is taken, Maximum stress and Tsai-wu.
Both of these two failure criteria are based on the stresses in the local axes
instead of principal normal stresses and maximum shear stresses, and four normal
strength parameters and one shear stress for a unidirectional lamina are
involved. The five strength parameters are

$(\sigma _1^{T})_{ult}= $ ultimate longitudinal tensile strength(in direction 1),

$(\sigma _1^{C})_{ult}= $ ultimate longitudinal compressive strength,

$(\sigma _2^{T})_{ult}= $ ultimate transverse tensile strength,

$(\sigma _2^{C})_{ult}= $ ultimate transverse compressive strength, and

$(\tau_{12})_{ult}= $ and ultimate in-plane shear strength.

\subsection{Maximum stress failure criterion}(MS)
Maximum stress failure theory consists of maximum normal stress theory proposed by Rankine and maximum 
shearing stress theory by Tresca. The stresses applied on a lamina can be resolved into the normal and shear stresses 
in the local axes. If any of the normal or shear stresses in the local axes of a lamina is equal or exceeds the corresponding 
ultimate strengths of the unidirectional lamina, the lamina is considered to be failed. That is,

$\sigma_1 \geq (\sigma _1^{T})_{ult} $ or $\sigma_1 \leq -(\sigma _1^{C})_{ult} $

$\sigma_2 \geq (\sigma _2^{T})_{ult} $ or $\sigma_2 \leq -(\sigma _2^{C})_{ult} $

$\tau_{12} \geq (\tau_{12})_{ult} $  or $\tau_{12} \leq -(\tau_{12})_{ult} $

where $\sigma_1$ and $\sigma_2$ are the normal stresses in the local axes 1 and 2, respectively;
$\tau_{12}$ is the shear stress in the symmetry plane 1-2.

\subsection{Tsai-wu failure criterion}
The TW criterion is one of the most reliable static failure criteria which is derived from the von
Mises yield criterion.  
A lamina is considered to fail
if \begin{equation} \label{eq:tsai_wu}
\begin{split}
	H_1 \sigma_1  & + H_2 \sigma_2 + H_6 \tau_{12} + H_{11}\sigma_1^2 + H_{22} \sigma_2^2 \\
				  & + H_{66}  \tau_{12}^2 + 2H_{12}\sigma_1\sigma_2 < 1
\end{split}
\end{equation}

is violated, where

\begin{equation}
	\begin{split}
		H_{1}&=\frac{1}{\left(\sigma_{1}^{T}\right)_{u l t}}-\frac{1}{\left(\sigma_{1}^{C}\right)_{u l t}} \\
		H_{11}&=\frac{1}{\left(\sigma_{1}^{T}\right)_{u l t}\left(\sigma_{1}^{C}\right)_{u l t}} \\
		H_{2}&=\frac{1}{\left(\sigma_{2}^{T}\right)_{u l t}}-\frac{1}{\left(\sigma_{2}^{C}\right)_{u l t}} \\
		H_{22}&=\frac{1}{\left(\sigma_{2}^{T}\right)_{u l t}\left(\sigma_{2}^{C}\right)_{u l t}} \\
		H_{66}&=\frac{1}{\left(\tau_{12}\right)_{u l t}^{2}} \\
		H_{12}&=-\frac{1}{2} \sqrt{\frac{1}{\left(\sigma_{1}^{T}\right)_{u l
		t}\left(\sigma_{1}^{C}\right)_{u l t}\left(\sigma_{2}^{T}\right)_{u l
		t}\left(\sigma_{2}^{C}\right)_{u l t}}}
	\end{split}
\end{equation}

$H_i$ is the strength tensors of the second order; $H_{ij}$ is the strength
tensors of the fourth order. $\sigma_1$ is the applied normal stress in 
direction 1; $\sigma_2$ is the applied normal stress in the direction 2; and
$\tau_{12}$ is the applied in-plane shear stress.





\begin{figure}
\centering
\begin{tikzpicture}
	\begin{scope}
		%\draw[style=help lines] (-3,-3) grid (3,3);
		\draw (0,0) rectangle (2,3);
		\draw[->] (1.3,1.2) -- (2.6,1.2);
		\draw[->] (1.3,1.2) -- (1.3,3.4);
		\node at (2.2,1) {$X_T$};
		\node at (1.5, 3.2) {$Y_T$};
		\node at (-0.2, 0.9) {$X_C$};
		\node at (1.8, -0.2) {$Y_C$};
	\end{scope}
	\begin{scope}[xshift=6cm,yshift=1.15cm]
		%\draw[style=help lines] (-3,-3) grid (3,3);
		\draw[rotate=30] (0,0) ellipse (2cm and 1cm);
		\draw[->] (0.2,0) -- (0.2,2.2);
		\draw[->] (0.2,0) -- (1.9,0);
		\node at (1.6,-0.2) {$X_T$};
		\node at (0.3, 1.3) {$Y_T$};
		\node at (-1.6, 0) {$X_C$};
		\node at (-0.5, -1.5) {$Y_C$};
	\end{scope}
\end{tikzpicture}
\caption{Schematic failure surfaces for maximum stress and quadratic failure
criteria}
\end{figure}


\subsection{Failure Theories for a Laminate}
If keep increasing the loading applied to a laminate, the laminate will fails. The failure process
of a laminate is more complicate than a lamina, because a laminate consists of multiple plies, and
the fiber orientation, material, thickness of each ply maybe different from the others. In most
situations, some layer fails first and the remains continue to take more loads until all the plies
fail.  If one ply fails, it means this lamina does not contribute to the load carrying capacity of
the laminate. The procedure for finding the first failure ply given follows the fully discounted
method:

\begin{enumerate}
\item Compute the reduced stiffness matrix [Q] referred to as the local axis for each ply using its
	four engineering elastic constants $E_1 $, $E_2 $, $E_{12} $, and $G_{12} $.

\item Calculate the transformed reduced stiffness $[\bar{Q}] $ referring to the global coordinate
	system (x, y) using the reduced stiffness matrix [Q] obtained in step 1 and the ply angle for
	each layer.

\item  Given the thickness and location of each layer, the three laminate stiffness matrices [A],
	[B], and [D] are determined.

\item  Apply the forces and moments, $[N]_{xy}, [M]_{xy} $ solve Equation
	\ref{eq:force_and_moments}, and calculate the middle plane strain $[\sigma ^{0}]_{xy} $ and
	curvature $[k]_{xy} $.

\item Determine the local strain and stress of each layer under the applied load.

\item  Use the ply-by-ply stresses and strains in the Tsai-wu failure theory to find the strength
	ratio, and the layer with smallest strenght ratio is the first failed ply. 
\end{enumerate}

\subsection{Safety factor}
The safety factor, or yield stress, is how much extra load beyond is intended a
composite laminate will actually take. The safey factor is defined as 

\begin{equation} \label{eq:sr}S R=\frac{\text {Maximum Load Which Can Be Applied}}{\text {Load Applied}}
\end{equation}.

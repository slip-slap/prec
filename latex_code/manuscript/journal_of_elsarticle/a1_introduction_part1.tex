\section{Introduction}
Fiber-reinforced composite materials have been widely used in a variety of
applications, which include electronic packaging, sports equipment,
homebuilding, medical prosthetic devices, high-performance military
structures, etc. because they offer improved mechanical stiffness, strength,
and low specific gravity of fibers over conventional materials.  The stacking
sequence, ply thickness, and fiber orientation of composite laminates give the
designer an additional ’degree of freedom’ to tailor the design with respect to
strength or stiffness. CLT and failure theory, e.g., Tsai-Wu failure criteria,
is usually taken to predict the behavior of a laminate from a knowledge of the
composite laminate properties of the individual layers and the laminate
geometry.

However, the use of CLT needs intensive computation which takes an analytical
method to solve the problem, since it involves massive matrix multiplication
and integration calculation. Techniques of function approximation can
accelerate the calculation process and reduce the computation cost.  Artificial
neural network(ANN), heavily inspired by biology and psychology, is a reliable
tool instead of a complicated mathematical model. ANN has been widely used to
solve various practical engineering problems in applications, such as pattern
recognition, nonlinear regression, data mining, clustering,  prediction, etc.
Evolutionary artificial neural networks(EANN's) is a special class of
artificial neural networks(ANN's), in which evolutionary algorithms are
introduced to design the topology of an ANN, and can be used at four different
levels: connection weights, architectures, input features, and learning rules.
It is shown that the combinations of ANN's and EA's can significantly improve
the performance of intelligent systems than that rely's on ANN's or
evolutionary algorithms alone.

The rest of the paper is organized as the following: chapter two explains the
classical laminate theory and the failure criteria taken in the present study;
chapter three explains the design of artifical neural network for mathmatical
model approximation; chaper four reviews the use of genetic algorithm in the
design of neural network architecture and the parameters optimization during
the training process of neural network design;  chapter five describes the result
of the numerical experiments in different cases; in the conclusion section
we dicuss the results.


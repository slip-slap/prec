\section{Experiment}
We applied this search strategy to dataset generated by the classic lamination
theory and failure theories. In this dataset, sixtheen attribues and two actual
values are given.

\subsection{Dataset Preparation}
Equation \ref{equ:stress-strain} takes an analytical approach to model the
relationship between stress and strain. We sample this function to yield 14000 points
uniformly distributed over the domain space.

The range of in-plane loading is from 0 to 120; the range of fiber orientation $\theta$ is from
-90 to 90; ply thickness $t$ is 1.27mm, number of plies range $N$ is from 4 to 120;
Three different material is used in this experiment, as shown in table \ref{tab:mat}.
Figure \ref{tab:traing-data} shows part of the training data.

In order to speeds up the learning and accerlate convergence, the input
atttributes of the data set are rescaled to between 0 and 1.0 by a linear function.

\input{a0_experiment_table_training_data}
\input{a0_table_composite_material}

\begin{figure}[h!]
	\centering
	\begin{subfigure}[b]{1.0\linewidth}
		\centering
		\includegraphics[width=0.8\textwidth]{./a0_figure_ann_for_clt_architecture_example1.png}
		\caption{Parent 1}
		\label{fig:p1}
	\end{subfigure}
	\newline
	\begin{subfigure}[b]{1.0\linewidth}
		\centering
		\includegraphics[width=0.8\textwidth]{./a0_figure_ann_for_clt_architecture_example2.png}
		\caption{Parent 2}
		\label{fig:p2}
	\end{subfigure}
	\newline
	\begin{subfigure}[b]{1.0\linewidth}
		\centering
		\includegraphics[width=0.8\textwidth]{./a0_figure_ann_for_clt_architecture_child.png}
		\caption{Child}
		\label{fig:child}
	\end{subfigure}
	\caption{Search Operation}
	\label{fig:search}
\end{figure}

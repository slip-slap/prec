\section{Experiment}
We applied this search strategy to dataset generated by the classic lamination
theory and failure theories. In this dataset, sixtheen attribues and two actual
values are given.

\subsection{Dataset Preparation}
Equation \ref{equ:stress-strain} takes an analytical approach to model the
relationship between stress and strain. We sample this function to yield 14000 points
uniformly distributed over the domain space.

The range of in-plane loading is from 0 to 120; the range of fiber orientation $\theta$ is from
-90 to 90; ply thickness $t$ is 1.27mm, number of plies range $N$ is from 4 to 120;
Three different material is used in this experiment, as shown in table \ref{tab:mat}.
Figure \ref{tab:traing-data} shows part of the training data.

In order to speeds up the learning and accerlate convergence, the input
atttributes of the data set are rescaled to between 0 and 1.0 by a linear function.

\begin{table}	
\label{tab:traing-data}
		\resizebox{\textwidth}{!}{
	\begin{tabular}{cccc|cc}
		\toprule
		\multicolumn{4}{c}{\textbf{Input}} &  \multicolumn{2}{c}{\textbf{Output}} \\
		\midrule
		Load  &  \makecell{Laminate \\ Structure }  & \makecell{Material \\ Property} & \makecell{Failure \\  Property}  & MS & Tsai-Wu \\
		\midrule

		-70,-10,-40,  & 90,-90,4,1.27, & 38.6,8.27,0.26,4.14,  & 1062.0,610.0,31,118,72,  & 0.0102, & 0.0086 \\
		-10,10,0,     & -86,86,80,1.27,& 181.0,10.3,0.28,7.17, & 1500.0,1500.0,40,246,68, & 0.4026, & 2.5120 \\
		-70,-50,80,   & -38,38,4,1.27, & 116.6,7.67,0.27,4.173,& 2062.0,1701.0,70,240,105,& 0.0080, & 0.0325 \\
		-70,80,-40,   & 90,-90,48,1.27,& 38.6,8.27,0.26,4.14,  & 1062.0,610.0,31,118,72,  & 0.0218, & 0.1028 \\
		-20,-30,0,    & -86,86,60,1.27,& 181.0,10.3,0.28,7.17, & 1500.0,1500.0,40,246,68, & 0.6481, & 0.9512 \\
		0,-40,0,      & 74,-74,168,1.27,& 181.0,10.3,0.28,7.17,& 1500.0,1500.0,40,246,68, & 1.3110, & 3.9619 \\
		\bottomrule
		\end{tabular}
	}
\end{table}

\begin{table*}[ht]
\caption{Comparison of the carbon/epoxy, graphite/epoxy, and glass/epoxy properties}
\centering
\begin{adjustbox}{width=1\textwidth}
\label{tab:mat}
\begin{tabular}{cccccc}
\toprule
Property								   & Symbol				  & Unit  &  Carbon/Epoxy&  Graphite/Epoxy  &  Glass/Epoxy   \\
\midrule
Longitudinal elastic modulus			   & $E_1$				  & GPa   &  116.6       &  181             &  38.6           \\
Traverse elastic modulus				   & $E_2$				  & GPa   &  7.67        &  10.3            &  8.27           \\
Major Poisson's ratio					   & $v_{12}$			  &       &  0.27        &  0.28            &  0.26           \\
Shear modulus							   & $G_{12}$			  & GPa   &  4.17        &  7.17            &  4.14           \\
Ultimate longitudinal tensile strength     & $(\sigma_1^T)_{ult}$ & MP    &  2062        &  1500            &  1062            \\
Ultimate longitudinal compressive strength & $(\sigma_1^C)_{ult}$ & MP    &  1701        &  1500            &  610             \\
Ultimate transverse tensile strength       & $(\sigma_2^T)_{ult}$ & MPa   &  70          &  40              &  31              \\
Ultimate transverse compressive strength   & $(\sigma_2^C)_{ult}$ & MPa   &  240         &  246             &  118              \\
Ultimate in-plane shear strength           & $(\tau_{12})_{ult}$  & MPa   &  105         &  68              &  72               \\
Density                                    & $\rho$               & $g/cm^3$ &  1.605    &  1.590           &  1.903               \\
Cost                                       &                      &       &  8           &  2.5             &  1               \\
\bottomrule
\end{tabular}
\end{adjustbox}
\end{table*}



%\begin{figure}
%\centering
\begin{tikzpicture}
	[ p/.style={ draw=none, fill=none, }, remember picture, 
	  net/.style={ matrix of nodes, nodes={ draw, circle, inner sep=7.5pt },
	  nodes in empty cells,
	  column sep=-10.5pt,
	  row sep=0.8cm
	  }
	]
%\draw[help lines] (-3cm,-6cm) grid (6cm,3cm);
\matrix[net] (mat)
{
		  & |[p]| &  & |[p]| &  & |[p]| &  & |[p]| &  & |[p]| &  & |[p]| &  & |[p]| &  & |[p]| &  &
			|[p]| &  & |[p]| &  & |[p]| &  & |[p]| &  & |[p]| &  & |[p]| &  & |[p]| &  & |[p]|    \\
	 |[p]| & |[p]| & |[p]| &  |[p]| &        & |[p]| & |[p]| & |[p]| &|[p]| &       & |[p]| &  |[p]| & |[p]| &
	 |[p]| &       & |[p]| &  |[p]| &  |[p]| & |[p]| & |[p]| &       &|[p]| & |[p]| & |[p]| & |[p]|
		   & |[p]| &       &  |[p]| &  |[p]| & |[p]| & |[p]| & |[p]| &|[p]| \\ 
	 |[p]| &  |[p]| & |[p]|  &  |[p]| & |[p]|  &  |[p]| &  |[p]| &  |[p]| & |[p]| & |[p]| & |[p]| &       & |[p]|
		   &  |[p]| & |[p]|  &  |[p]| &        &  |[p]| &  |[p]| &  |[p]| & |[p]| & |[p]| & |[p]| & |[p]| &     |[p]|
		   &  |[p]| & |[p]|  &  |[p]| & |[p]|  &  |[p]| &  |[p]| &  |[p]| \\ 
	  };
	  \node at (mat-1-1.base)  {$i_1$};
	  \node at (mat-1-3.base)  {$i_2$};
	  \node at (mat-1-5.base)  {$i_3$};
	  \node at (mat-1-7.base)  {$i_4$};
	  \node at (mat-1-9.base)  {$i_5$};
	  \node at (mat-1-11.base)  {$i_6$};
	  \node at (mat-1-13.base)  {$i_7$};
	  \node at (mat-1-15.base)  {$i_8$};
	  \node at (mat-1-17.base)  {$i_9$};
	  \node at (mat-1-17.base)  {$i_9$};
	  \node at (mat-1-19.base)  {$i_{10}$};
	  \node at (mat-1-21.base)  {$i_{11}$};
	  \node at (mat-1-23.base)  {$i_{12}$};
	  \node at (mat-1-25.base)  {$i_{13}$};
	  \node at (mat-1-27.base)  {$i_{14}$};
	  \node at (mat-1-29.base)  {$i_{15}$};
	  \node at (mat-1-31.base)  {$i_{16}$};

	  \node at (mat-2-5.base)  {$h_1$};
	  \node at (mat-2-10.base) {$h_2$};
	  \node at (mat-2-15.base) {$h_3$};
	  \node at (mat-2-21.base) {$h_4$};
	  \node at (mat-2-27.base) {$h_5$};
		 \foreach \a in {1,3,5,7,9,11,31}{
				\draw[->,dotted] (mat-1-\a.south) -- (mat-2-5.north);
			 }
		 \foreach \a in {5,7,11,13,19,25,27}{
				\draw[->,dotted] (mat-1-\a.south) -- (mat-2-10.north);
			 }
		 \foreach \a in {1,7,11,13,17,19,25}{
				\draw[->,dotted] (mat-1-\a.south) -- (mat-2-15.north);
			 }
		 \foreach \a in {5,9,19,21,29}{
				\draw[->,dotted] (mat-1-\a.south) -- (mat-2-21.north);
			 }
		 \foreach \a in {11,15,19,23,27,29,31}{
				\draw[->,dotted] (mat-1-\a.south) -- (mat-2-27.north);
			 }
		 \foreach \c in {5,10,15,21,27}{
			\foreach \d in {12,17}{
				\draw[->,dotted] (mat-2-\c.south) -- (mat-3-\d.north);
			}
 }
\end{tikzpicture}
%\caption{Parent 1\label{fig:parents1}}
%\end{figure}

\begin{figure*}
\centering
\begin{tikzpicture}
[ p/.style={ draw=none, fill=none, }, remember picture, 
  net/.style={ matrix of nodes, nodes={ draw, circle, inner sep=7.5pt },
  nodes in empty cells,
  column sep=-10.5pt,
  row sep=0.8cm
  }
]
%\draw[help lines] (-3cm,-6cm) grid (6cm,3cm);
\matrix[net] (mat)
{
	  & |[p]| &  & |[p]| &  & |[p]| &  & |[p]| &  & |[p]| &  & |[p]| &  & |[p]| &  & |[p]| &  &
	    |[p]| &  & |[p]| &  & |[p]| &  & |[p]| &  & |[p]| &  & |[p]| &  & |[p]| &  & |[p]|    \\
 |[p]| &  |[p]|& |[p]| &        &  |[p]| &       & |[p]| &       &|[p]| &       & |[p]| &   & |[p]| &
       &  |[p]|&       &  |[p]| &       & |[p]| &       &|[p]| &       & |[p]| &  
	   & |[p]| &       &  |[p]| &  |[p]| & |[p]| & |[p]| & |[p]| &|[p]| & |[p]| \\ 
 |[p]| &  |[p]| & |[p]|  &  |[p]| & |[p]|  &  |[p]| &  |[p]| &  |[p]| & |[p]| & |[p]| & |[p]| &       & |[p]|
	   &  |[p]| & |[p]|  &  |[p]| &        &  |[p]| &  |[p]| &  |[p]| & |[p]| & |[p]| & |[p]| & |[p]| &     |[p]|
	   &  |[p]| & |[p]|  &  |[p]| & |[p]|  &  |[p]| &  |[p]| &  |[p]| \\ 
  };
  \node at (mat-1-1.base)  {$i_1$};
  \node at (mat-1-3.base)  {$i_2$};
  \node at (mat-1-5.base)  {$i_3$};
  \node at (mat-1-7.base)  {$i_4$};
  \node at (mat-1-9.base)  {$i_5$};
  \node at (mat-1-11.base)  {$i_6$};
  \node at (mat-1-13.base)  {$i_7$};
  \node at (mat-1-15.base)  {$i_8$};
  \node at (mat-1-17.base)  {$i_9$};
  \node at (mat-1-17.base)  {$i_9$};
  \node at (mat-1-19.base)  {$i_{10}$};
  \node at (mat-1-21.base)  {$i_{11}$};
  \node at (mat-1-23.base)  {$i_{12}$};
  \node at (mat-1-25.base)  {$i_{13}$};
  \node at (mat-1-27.base)  {$i_{14}$};
  \node at (mat-1-29.base)  {$i_{15}$};
  \node at (mat-1-31.base)  {$i_{16}$};

  \node at (mat-2-5.base)  {$h_1$};
  \node at (mat-2-10.base) {$h_2$};
  \node at (mat-2-15.base) {$h_3$};
  \node at (mat-2-21.base) {$h_4$};
  \node at (mat-2-27.base) {$h_5$};
 \foreach \a in {13,15,17,19,21}{
        \draw[->] (mat-1-\a.south) -- (mat-2-4.north);
     }
 \foreach \a in {1,3,5,7}{
        \draw[->] (mat-1-\a.south) -- (mat-2-6.north);
     }
 \foreach \a in {1,3,5,7,17,19,21,23}{
        \draw[->] (mat-1-\a.south) -- (mat-2-8.north);
     }
 \foreach \a in {5,9,11,13,15}{
        \draw[->] (mat-1-\a.south) -- (mat-2-10.north);
     }
 \foreach \a in {23,27,29,31}{
        \draw[->] (mat-1-\a.south) -- (mat-2-12.north);
     }
 \foreach \a in {11,15,19}{
        \draw[->] (mat-1-\a.south) -- (mat-2-14.north);
     }
 \foreach \a in {27,29,31}{
        \draw[->] (mat-1-\a.south) -- (mat-2-16.north);
     }
 \foreach \a in {11,19,27,31}{
        \draw[->] (mat-1-\a.south) -- (mat-2-18.north);
     }
 \foreach \a in {15,19,23}{
        \draw[->] (mat-1-\a.south) -- (mat-2-20.north);
     }
 \foreach \a in {3,5,7}{
        \draw[->] (mat-1-\a.south) -- (mat-2-22.north);
     }
 \foreach \a in {17,19,21,23}{
        \draw[->] (mat-1-\a.south) -- (mat-2-24.north);
     }
 \foreach \a in {21,23,25,27,29}{
        \draw[->] (mat-1-\a.south) -- (mat-2-26.north);
     }
 \foreach \c in {4,6,8,10,12,14,16,18,20,22,24,26}{
    \foreach \d in {12,17}{
 		\draw[->] (mat-2-\c.south) -- (mat-3-\d.north);
	}
 }

\end{tikzpicture}
\caption{Neural Network Model}
\end{figure*}


\begin{table}[h!]
	\centering
		\caption{first table}
		\begin{tabular}{l|cccc cccc cccc cccc}
			\toprule
hi    & $i_1$ & $i_2$ & $i_3$ & $i_4$ & $i_5$ & $i_6$ & $i_7$ & $i_8$ & $i_9$ & $i_{10}$ & $i_{11}$ & $i_{12}$ & $i_{13}$ & $i_{14}$ & $i_{15}$ & $i_{16}$ \\
			\midrule
$h_1$ & 1  & 1 & 1  & 1  & 1 & 1 & 0 & 0  & 0 & 0 & 0 & 0  & 0 & 0  & 1 & 1 \\
$h_2$ & 0  & 1 & 1  & 1  & 0 & 0 & 0 & 1  & 0 & 0 & 1 & 1  & 0 & 0  & 0 & 0 \\
$h_3$ & 1  & 0 & 0  & 1  & 0 & 1 & 1 & 0  & 1 & 1 & 0 & 0  & 1 & 0  & 0 & 0 \\
$h_4$ & 0  & 0 & 1  & 0  & 1 & 0 & 0 & 0  & 0 & 1 & 0 & 1  & 0 & 0  & 1 & 0 \\
$h_5$ & 0  & 0 & 0  & 0  & 0 & 1 & 0 & 1  & 0 & 1 & 0 & 1  & 0 & 1  & 1 & 1 \\
			\bottomrule
		\end{tabular}
\end{table}


\begin{figure*}
\centering
\begin{tikzpicture}
[ p/.style={ draw=none, fill=none, }, remember picture, 
  net/.style={ matrix of nodes, nodes={ draw, circle, inner sep=7.5pt },
  nodes in empty cells,
  column sep=-10.5pt,
  row sep=0.8cm
  }
]
%\draw[help lines] (-3cm,-6cm) grid (6cm,3cm);
\matrix[net] (mat)
{
	  & |[p]| &  & |[p]| &  & |[p]| &  & |[p]| &  & |[p]| &  & |[p]| &  & |[p]| &  & |[p]| &  &
	    |[p]| &  & |[p]| &  & |[p]| &  & |[p]| &  & |[p]| &  & |[p]| &  & |[p]| &  & |[p]|    \\
 |[p]| & |[p]| & |[p]|   & |[p]|  &  |[p]| &  |[p]| & |[p]| &|[p]|  &   & |[p]| &      & |[p]| &  &  |[p]| &  &
 |[p]| &       & |[p]| &   & |[p]|   &  & |[p]| &       &|[p]| & |[p]| & |[p]| & |[p]|
	   & |[p]| &  |[p]|   &  |[p]| &  |[p]| & |[p]| & |[p]|  \\ 
 |[p]| &  |[p]| & |[p]|  &  |[p]| & |[p]|  &  |[p]| &  |[p]| &  |[p]| & |[p]| & |[p]| & |[p]| &       & |[p]|
	   &  |[p]| & |[p]|  &  |[p]| &        &  |[p]| &  |[p]| &  |[p]| & |[p]| & |[p]| & |[p]| & |[p]| &     |[p]|
	   &  |[p]| & |[p]|  &  |[p]| & |[p]|  &  |[p]| &  |[p]| &  |[p]| \\ 
  };
  \node at (mat-1-1.base)   {$i_1$};    \node at (mat-1-3.base)   {$i_2$}; 
  \node at (mat-1-5.base)   {$i_3$};    \node at (mat-1-7.base)   {$i_4$}; 
  \node at (mat-1-9.base)   {$i_5$};    \node at (mat-1-11.base)  {$i_6$}; 
  \node at (mat-1-13.base)  {$i_7$};    \node at (mat-1-15.base)  {$i_8$}; 
  \node at (mat-1-17.base)  {$i_9$};    \node at (mat-1-17.base)  {$i_9$};
  \node at (mat-1-19.base)  {$i_{10}$}; \node at (mat-1-21.base)  {$i_{11}$};
  \node at (mat-1-23.base)  {$i_{12}$}; \node at (mat-1-25.base)  {$i_{13}$};
  \node at (mat-1-27.base)  {$i_{14}$}; \node at (mat-1-29.base)  {$i_{15}$};
  \node at (mat-1-31.base)  {$i_{16}$};

  \node at (mat-2-9.base)  {$h_1$};
  \node at (mat-2-11.base) {$h_2$};
  \node at (mat-2-13.base) {$h_3$};
  \node at (mat-2-15.base) {$h_4$};
  \node at (mat-2-17.base) {$h_5$};
  \node at (mat-2-19.base) {$h_6$};
  \node at (mat-2-21.base) {$h_7$};
  \node at (mat-2-23.base) {$h_8$};

 \foreach \a in {1,3,5,7,9,11,31}{
        \draw[->,dotted] (mat-1-\a.south) -- (mat-2-9.north);
     }
 \foreach \a in {5,7,11,13,19,25,27}{
        \draw[->,dotted] (mat-1-\a.south) -- (mat-2-11.north);
     }
 \foreach \a in {27,29,31}{
        \draw[->,dashed] (mat-1-\a.south) -- (mat-2-13.north);
     }
 \foreach \a in {11,19,27,31}{
        \draw[->,dashed] (mat-1-\a.south) -- (mat-2-15.north);
     }
 \foreach \a in {15,19,23}{
        \draw[->,dashed] (mat-1-\a.south) -- (mat-2-17.north);
     }
 \foreach \a in {3,5,7}{
        \draw[->,dashed] (mat-1-\a.south) -- (mat-2-19.north);
     }
 \foreach \a in {17,19,21,23}{
        \draw[->,dashed] (mat-1-\a.south) -- (mat-2-21.north);
     }
 \foreach \a in {21,23,25,27,29}{
        \draw[->,dashed] (mat-1-\a.south) -- (mat-2-23.north);
     }

 \foreach \c in {9,11,13,15,17,19,21,23}{
    \foreach \d in {12,17}{
 		\draw[->] (mat-2-\c.south) -- (mat-3-\d.north);
	}
 }

\end{tikzpicture}
\caption{Child}
\end{figure*}





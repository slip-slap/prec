\section{Experiment}
\subsection{Train Data Preparation}
Equation \ref{equ:stress-strain} takes an analytical approach to model the
relationship between stress and strain. We sample this function to yield 14000 points
uniformly distributed over the domain space.

The range of in-plane loading is from 0 to 120; fiber orientation range is from
-90 to 90; ply thickness is 1.27mm, number of plies range N is from 4 to 120;
Three different material is used in this experiment, as shown in table \ref{tab:mat}.

In order to speeds up the learning and accerlate convergence, the
training data is normalized through a min-max normalization trick.

\input{a0_table_composite_material}

Figure \ref{tab:traing-data} shows part of the training data.

\input{a0_experiment_table_training_data}



\input{a0_figure_ann_for_clt_architecture_example1}
\begin{figure*}
\centering
\begin{tikzpicture}
	[ p/.style={ draw=none, fill=none, }, remember picture, 
	  net/.style={ matrix of nodes, nodes={ draw, circle, inner sep=7.5pt },
	  nodes in empty cells,
	  column sep=-10.5pt,
	  row sep=0.8cm
	  }
	]
%\draw[help lines] (-3cm,-6cm) grid (6cm,3cm);
\matrix[net] (mat)
	{
		  & |[p]| &  & |[p]| &  & |[p]| &  & |[p]| &  & |[p]| &  & |[p]| &  & |[p]| &  & |[p]| &  &
			|[p]| &  & |[p]| &  & |[p]| &  & |[p]| &  & |[p]| &  & |[p]| &  & |[p]| &  & |[p]|    \\
	 |[p]| &  |[p]|& |[p]| &        &  |[p]| &       & |[p]| &       &|[p]| &       & |[p]| &   & |[p]| &
		   &  |[p]|&       &  |[p]| &       & |[p]| &       &|[p]| &       & |[p]| &  
		   & |[p]| &       &  |[p]| &  |[p]| & |[p]| & |[p]| & |[p]| &|[p]| & |[p]| \\ 
	 |[p]| &  |[p]| & |[p]|  &  |[p]| & |[p]|  &  |[p]| &  |[p]| &  |[p]| & |[p]| & |[p]| & |[p]| &       & |[p]|
		   &  |[p]| & |[p]|  &  |[p]| &        &  |[p]| &  |[p]| &  |[p]| & |[p]| & |[p]| & |[p]| & |[p]| &     |[p]|
		   &  |[p]| & |[p]|  &  |[p]| & |[p]|  &  |[p]| &  |[p]| &  |[p]| \\ 
	  };
	  \node at (mat-1-1.base)  {$i_1$};
	  \node at (mat-1-3.base)  {$i_2$};
	  \node at (mat-1-5.base)  {$i_3$};
	  \node at (mat-1-7.base)  {$i_4$};
	  \node at (mat-1-9.base)  {$i_5$};
	  \node at (mat-1-11.base)  {$i_6$};
	  \node at (mat-1-13.base)  {$i_7$};
	  \node at (mat-1-15.base)  {$i_8$};
	  \node at (mat-1-17.base)  {$i_9$};
	  \node at (mat-1-17.base)  {$i_9$};
	  \node at (mat-1-19.base)  {$i_{10}$};
	  \node at (mat-1-21.base)  {$i_{11}$};
	  \node at (mat-1-23.base)  {$i_{12}$};
	  \node at (mat-1-25.base)  {$i_{13}$};
	  \node at (mat-1-27.base)  {$i_{14}$};
	  \node at (mat-1-29.base)  {$i_{15}$};
	  \node at (mat-1-31.base)  {$i_{16}$};

	  \node at (mat-2-5.base)  {$h_1$};
	  \node at (mat-2-10.base) {$h_2$};
	  \node at (mat-2-15.base) {$h_3$};
	  \node at (mat-2-21.base) {$h_4$};
	  \node at (mat-2-27.base) {$h_5$};
	 \foreach \a in {13,15,17,19,21}{
			\draw[->, dashed] (mat-1-\a.south) -- (mat-2-4.north);
		 }
	 \foreach \a in {1,3,5,7}{
			\draw[->, dashed] (mat-1-\a.south) -- (mat-2-6.north);
		 }
	 \foreach \a in {1,3,5,7,17,19,21,23}{
			\draw[->, dashed] (mat-1-\a.south) -- (mat-2-8.north);
		 }
	 \foreach \a in {5,9,11,13,15}{
			\draw[->, dashed] (mat-1-\a.south) -- (mat-2-10.north);
		 }
	 \foreach \a in {23,27,29,31}{
			\draw[->, dashed] (mat-1-\a.south) -- (mat-2-12.north);
		 }
	 \foreach \a in {11,15,19}{
			\draw[->, dashed] (mat-1-\a.south) -- (mat-2-14.north);
		 }
	 \foreach \a in {27,29,31}{
			\draw[->, dashed] (mat-1-\a.south) -- (mat-2-16.north);
		 }
	 \foreach \a in {11,19,27,31}{
			\draw[->, dashed] (mat-1-\a.south) -- (mat-2-18.north);
		 }
	 \foreach \a in {15,19,23}{
			\draw[->, dashed] (mat-1-\a.south) -- (mat-2-20.north);
		 }
	 \foreach \a in {3,5,7}{
			\draw[->, dashed] (mat-1-\a.south) -- (mat-2-22.north);
		 }
	 \foreach \a in {17,19,21,23}{
			\draw[->, dashed] (mat-1-\a.south) -- (mat-2-24.north);
		 }
	 \foreach \a in {21,23,25,27,29}{
			\draw[->, dashed] (mat-1-\a.south) -- (mat-2-26.north);
		 }
	 \foreach \c in {4,6,8,10,12,14,16,18,20,22,24,26}{
		\foreach \d in {12,17}{
			\draw[->, dashed] (mat-2-\c.south) -- (mat-3-\d.north);
		}
	 }
\end{tikzpicture}
\caption{Parent 2}
\end{figure*}




\input{a0_figure_ann_for_clt_architecture_child}

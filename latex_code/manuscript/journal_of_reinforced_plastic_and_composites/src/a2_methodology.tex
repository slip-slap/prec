\section{Methodology}
\subsection{Objective function}
The optimization problem can be formulated as searching the optimal stacking
sequence of composite laminate.  There are two design variables here, the angles
in the laminate, and the number of layers that each fiber orientation has. The
objective function is formulated as

\begin{equation}
	\begin{split}
    	F  &= 2t_0 \sum_{k=1}^n n_k  \textstyle{,}\\
    	   &SF_{MS} \geq 1 \textstyle{,} \\
    	   &SF_{TW} \geq 1 \textstyle{.}
	\end{split} 
\end{equation}

The first term represents the total thickness of the composite laminates, $t_0$
is the ply thickness; $n_k$ is the number of plies in the kth lamina, in which
the fiber orientation is $\theta_k$. The constraints here are two safety
factors should not less than 1, which means  $SF_{MS} \geq 1$, and $SF_{TW} \geq
1$, respectively.

\subsection{Encoding}
Due to the simplicity and efficiency of float representation, this encoding
method is implemented to represent a possible solution. As shown in Figure \ref{GA:operator}
 (a), these two chromsomes represent a $[+8_{7}/-9_{2}]_s$
carbon T300/5308 laminated composite, and $[+19_{4}/-36_{6}]_s$, respectively.
Becasue the laminate adopted in this paper is symmetric to its mid-plane, so
only half needs to be encoded.

\subsection{Selection}
The purpose of the selection operator is to chose mating pool to produce
alternative solutions of better fitness. Traditional methods of selecting
strategies only take the fitness of individuals into acount, however, due to 
the existance of constraint, various selection schemes are implemented to
selecet the mating set. Based on different selection schemes, the parents of
next generation can be divided into  three groups: proper groups, active groups,
and potential groups according to different selecting methods. 

Proper parents mean in which individual fullfils the constraints, which are
chosen by the individual's fitnees, individuals with better fitness are more
likely to be chosen if they fit the constraint; active groups means that
individual is supposed to be always exist in the parents during the GA, which
are selected by fitness, ignoring the constraint; The individuals from active
group may not correspond to feasible solutions, but their existance enriches the
variety of the gene clips.  Potential groups means that they are likely to turn
into proper individual after a couple of generations, and potential individuals
are chosen by constraint function, the more the individual fulfils the
constraint, the more possiblity it will be selected.

\subsection{Crossover}
The crossover operator happens among these three groups. the child of two proper
groups are more likely to be a proper individual which can be used to obtain a
alternative feasible solution. the child of an active individual and a potential
individual can significantly change the gene of active individual's chromsome,
which makes the individual evolve toward a new direction. The offspring of two
active individuals are more likely to be an active individual, which can maitain
the active group.  The figure.\ref{GA:operator} (b) shows two children $O_1$
and $O_2$ from two parents $P_1$ and $P_2$, each angle $C_a$ and it's length 
$C_l$ of a child are obtained by the following formula
\begin{align} 
	\begin{cases}
	C_a &= (P1_a + P2_a)/2 \\
	C_l &= (P1_l + P2_l)/2 
\end{cases} \textstyle{.}
\end{align}
\begin{table*}[!ht]
\caption{Properties of T300/5308 carbon/epoxy composite}
\label{tab:T300/5308}
\begin{adjustbox}{width=1\textwidth}
\centering
\begin{tabularx}{0.8\textwidth}{p{0.4\textwidth}p{0.1\textwidth}p{0.1\textwidth}c}
	\toprule
	\textbf{Property}								   & \textbf{Symbol}				  & \textbf{Unit}  & \textbf{ Graphite/Epoxy}     \\
	\midrule
	Longitudinal elastic modulus			   & $E_1$				  & GPa   &  181                 \\
	Traverse elastic modulus				   & $E_2$				  & GPa   &  10.3                \\
	Major Poisson's ratio					   & $v_{12}$			  &       &  0.28                \\
	Shear modulus							   & $G_{12}$			  & GPa   &  7.17                \\
	Ultimate longitudinal tensile strength     & $(\sigma_1^T)_{ult}$ & MP    &  1500                 \\
	Ultimate longitudinal compressive strength & $(\sigma_1^C)_{ult}$ & MP    &  1500                 \\
	Ultimate transverse tensile strength       & $(\sigma_2^T)_{ult}$ & MPa   &  40                   \\
	Ultimate transverse compressive strength   & $(\sigma_2^C)_{ult}$ & MPa   &  246                   \\
	Ultimate in-plane shear strength           & $(\tau_{12})_{ult}$  & MPa   &  68                    \\
	\bottomrule
\end{tabularx}
\end{adjustbox}
\end{table*}


\subsection{Mutation}
A mutation direction is imposed on the mutation operator which to make sure the
individual evolving toward the right direction. The mutation direction, denoted
by $md$, is a n dimensional vector corresonding to the number of constraints, it
is decided by the constraint thresholds $CT_i$ and the current individual's
constraint value, denoted as $CV_i$,  The mutation vector can be obtained by the
following formula

\resizebox{.9\linewidth}{!}{
	$\text{md} = [CT_1, \cdots, CT_{n-1}, CT_n] -  [CV_0, \cdots, CV_{n-1}, CV_n]$
}

During the operator, the mutation procedure is consist of two phases: the length
mutation of the chromsome, and the angle mutation of the chromsome.  Becasue the
chromsome's length is positive correlated with the individual's fitness, the
coefficient of length mutation denoted by $C_l$, if $\sum_{i=1}^{N}{CT_i}$ great
than $\sum_{i=1}^{N}{CV_i}$ , the mutation length is restricted to the range
$[0,(C_l \sum_{i=1}^{N}{(CT_i-CV_i)})/N]$, which means increase the chromsome's
length;Assuming a $[+13_6/-27_4]_s$ T300/5308 carbon/epoxy composite
laminate under the loading $N_{x} = N_{y} = 10$ MPa m, it's property as shown
in table \ref{tab:T300/5308}. According to CLT and
failure theory, the two safety factors $SF_{MS}$ and $SF_{TW}$ are  0.0539, and
0.0540, respectively. So the mutation vector and is $[0.9461,0.9460]$, assuming
the length mutation coefficient is 20, so the mutation range is from 0 to 18. A
random number is generated from the range $[0, 18]$, supposing the outcome is
13, then a length generator is used to a list, the it's sum is 13, suppose the
list is [5, 8], the laminate after mutation is $[13_{11}/-27_{12}]_s$.

If the $\sum_{i=1}^{N}{CT_i}$ less than $\sum_{i=1}^{N}{CV_i}$, the
mutation length is restricted to the range $([\sum_{i=1}^{N}{CT_i-CV_i})/N,0]$,
which means the individual's fitness exceeds the threshold value, and decrease
the chromsome's length.  Assuming a $[+33_{35}/-29_{26}]_s$ T300/5308 laminate
is under loading $N_{x}=10$ MPa, and $N_{y}=5$ MPa, then, it's $SF_{MS}$
constraint and $SF_{TW}$ values are  1.0912, 1.0747, respectively.
because the length mutation is 20, so the mutation range is from -2 to 0. This
would decrease the chromsome's length. 
\begin{align}
	\text{LM} = 
	\begin{cases}
		\resizebox{.35\textwidth}{!}{$[0,(C_l \sum_{i=1}^{N}{(CT_i-CV_i)})/N], \text{ if }  \sum_{i=1}^{N}{CT_i} > 
		\sum_{i=1}^{N}{CV_i}$}\\
		\resizebox{.35\textwidth}{!}{$[(C_l \sum_{i=1}^{N}{(CT_i-CV_i)})/N,0], \text{ if }  \sum_{i=1}^{N}{CT_i} < 
		\sum_{i=1}^{N}{CV_i}$}\\
	\end{cases} 
\end{align}

The relationship between the angles in the composite laminate and the
chromsome's fitness is unclear, so the mutation direction of chromsome's angle
is random. The coefficient angle mutation is $C_a$, the angle mutation range is
$[0,C_a \sum_{i=1}^{N}{(|CT_i-CV_i|)}]$ or $[C_a
\sum_{i=1}^{N}{(-|CT_i-CV_i|)},0]$. It is can be written as

\begin{align}
	\text{P(AM)} =  
	\begin{cases}
		0.5, \text{ AM = }[0,C_a \sum_{i=1}^{N}{(|CT_i-CV_i|)}] \\ 
	    0.5, \text{ AM = }[C_a \sum_{i=1}^{N}{(-|CT_i-CV_i|)},0]
	\end{cases} \textstyle{.}
\end{align}

\begin{figure}[!htb]
\setlength{\fboxsep}{0pt}%
\setlength{\fboxrule}{0pt}%
\begin{center}
\resizebox{.95\linewidth}{!}{
		\begin{tikzpicture}
		\tikzstyle{rec} = [rectangle, minimum width=0.8cm,minimum height=0.6cm, text
		centered, draw=black]
			\node (gene11) [rec] {90};
			\node (gene2) [rec] at ($(gene11.east)+(0.4cm,0)$)  {90};
			\node (gene3) [rec] at ($(gene2.east)+(0.4cm,0)$)  {0};
			\node (gene4) [rec] at ($(gene3.east)+(0.4cm,0)$)  {0};
			\node (gene5) [rec] at ($(gene4.east)+(0.4cm,0)$)  {0};
			\node (gene6) [rec] at ($(gene5.east)+(0.4cm,0)$)  {90};
			\node (gene7) [rec] at ($(gene6.east)+(0.4cm,0)$)  {90};
			\node (gene8) [rec] at ($(gene7.east)+(0.4cm,0)$)  {90};
			\node (gene9) [rec] at ($(gene8.east)+(0.4cm,0)$)  {90};
			\node (last) [rec] at ($(gene9.east)+(0.4cm,0)$)  {90};
			\node[text width=1cm] at ($(gene11.west)+(-0.3cm,0)$) {$P_1$:};
			\node (gene1) [rec] at ($(gene11.east)+(-0.4cm,-0.8cm)$) {0};
			\node (gene2) [rec] at ($(gene1.east)+(0.4cm,0)$)  {0};
			\node (gene3) [rec] at ($(gene2.east)+(0.4cm,0)$)  {90};
			\node (gene4) [rec] at ($(gene3.east)+(0.4cm,0)$)  {90};
			\node (gene5) [rec] at ($(gene4.east)+(0.4cm,0)$)  {90};
			\node (gene6) [rec] at ($(gene5.east)+(0.4cm,0)$)  {0};
			\node (gene7) [rec] at ($(gene6.east)+(0.4cm,0)$)  {0};
			\node (gene8) [rec] at ($(gene7.east)+(0.4cm,0)$)  {90};
			\node (gene9) [rec] at ($(gene8.east)+(0.4cm,0)$)  {0};
			\node (gene10) [rec] at ($(gene9.east)+(0.4cm,0)$)  {0};
			\node[text width=1cm] at ($(gene1.west)+(-0.3cm,0)$) {$P_2$:};
			\draw[-,white] ($(gene10.north)$)-- ++(0,-1.5cm);
			\node (label1) at ($(gene5.south)+(0cm,-0.5cm)$) {(a): Parents $P_1$ and $P_2$};
		\end{tikzpicture}
}

% offspring
\resizebox{.95\linewidth}{!}{
		\begin{tikzpicture}
			\tikzstyle{rec} = [rectangle, minimum width=0.8cm,minimum height=0.6cm, text
			centered, draw=black]
			\node (gene11) [rec] {90};
			\node (gene2) [rec] at ($(gene11.east)+(0.4cm,0)$) {90};
			\node (gene3) [rec] at ($(gene2.east)+(0.4cm,0)$)  {0};
			\node (gene4) [rec] at ($(gene3.east)+(0.4cm,0)$)  {0};
			\node (gene5) [rec] at ($(gene4.east)+(0.4cm,0)$)  {0};
			\node (gene6) [rec] at ($(gene5.east)+(0.4cm,0)$)  {0};
			\node (gene7) [rec] at ($(gene6.east)+(0.4cm,0)$)  {0};
			\node (gene8) [rec] at ($(gene7.east)+(0.4cm,0)$)  {90};
			\node (gene9) [rec] at ($(gene8.east)+(0.4cm,0)$)  {0};
			\node (last) [rec] at ($(gene9.east)+(0.4cm,0)$)  {0};
			\node[text width=1cm] at ($(gene11.west)+(-0.3cm,0)$) {$O_1$:};
			\node (gene1) [rec] at ($(gene11.east)+(-0.4cm,-0.8cm)$) {0};
			\node (gene2) [rec] at ($(gene1.east)+(0.4cm,0)$)  {0};
			\node (gene3) [rec] at ($(gene2.east)+(0.4cm,0)$)  {90};
			\node (gene4) [rec] at ($(gene3.east)+(0.4cm,0)$)  {90};
			\node (gene5) [rec] at ($(gene4.east)+(0.4cm,0)$)  {90};
			\node (gene6) [rec] at ($(gene5.east)+(0.4cm,0)$)  {90};
			\node (gene7) [rec] at ($(gene6.east)+(0.4cm,0)$)  {90};
			\node (gene8) [rec] at ($(gene7.east)+(0.4cm,0)$)  {90};
			\node (gene9) [rec] at ($(gene8.east)+(0.4cm,0)$)  {90};
			\node (gene10) [rec] at ($(gene9.east)+(0.4cm,0)$)  {90};
			\node[text width=1cm] at ($(gene1.west)+(-0.3cm,0)$) {$O_2$:};
			\draw[-,white] ($(gene10.north)$)-- ++(0,-1.5cm);
			\node (label1) at ($(gene5.south)+(0cm,-0.5cm)$) {(b): Offspring $O_1$ and $O_2$};
		\end{tikzpicture}
}

%mutation
\resizebox{.95\linewidth}{!}{
	\begin{tikzpicture}
	\tikzstyle{rec} = [rectangle, minimum width=0.8cm,minimum height=0.6cm, text
	centered, draw=black]
		\tikzstyle{rec} = [rectangle, minimum width=0.8cm,minimum height=0.6cm, text
		centered, draw=black]
		%\draw[help lines](-3,-3) grid (4,4);
		\node (gene11) [rec] {90};
		\node (gene2) [rec] at ($(gene11.east)+(0.4cm,0)$)  {90};
		\node (gene3) [rec] at ($(gene2.east)+(0.4cm,0)$)  {90};
		\node (gene4) [rec] at ($(gene3.east)+(0.4cm,0)$)  {$\cdots$};
		\node (gene5) [rec] at ($(gene4.east)+(0.4cm,0)$)  {90};
		\node (gene6) [rec] at ($(gene5.east)+(0.4cm,0)$)  {90};
		\node (gene7) [rec] at ($(gene6.east)+(0.4cm,0)$)  {0};
		\node (gene8) [rec] at ($(gene7.east)+(0.4cm,0)$)  {$\cdots$};
		\node (gene9) [rec] at ($(gene8.east)+(0.4cm,0)$)  {0};
		\node (last) [rec] at ($(gene9.east)+(0.4cm,0)$)  {0};
		\draw[<->,thick] (gene11.south) .. controls +(1.8,-0.4) .. (gene6.south)
			node[pos=0.5] {10} ;
		\draw[<->,thick] (gene7.south) .. controls +(1.3,-0.4) .. (last.south)
			node[pos=0.5] {9};
		\node[text width=1cm] at ($(gene11.west)+(-0.3cm,0)$) {$O_1$:};

		\node (label1) at ($(gene5.south)+(0cm,-0.8cm)$) {(c): Offspring $O_1$ after
			lenght mutation};

		\node (gene1) [rec] at ($(gene11.east)+(-0.4cm,-1.8cm)$) {90};
		\node (gene2) [rec] at ($(gene1.east)+(0.4cm,0)$)  {90};
		\node (gene3) [rec] at ($(gene2.east)+(0.4cm,0)$)  {90};
		\node (gene4) [rec] at ($(gene3.east)+(0.4cm,0)$)  {$\cdots$};
		\node (gene5) [rec] at ($(gene4.east)+(0.4cm,0)$)  {90};
		\node (gene6) [rec] at ($(gene5.east)+(0.4cm,0)$)  {0};
		\node (gene7) [rec] at ($(gene6.east)+(0.4cm,0)$)  {0};
		\node (gene8) [rec] at ($(gene7.east)+(0.4cm,0)$)  {$\cdots$};
		\node (gene9) [rec] at ($(gene8.east)+(0.4cm,0)$)  {0};
		\node (last) [rec] at ($(gene9.east)+(0.4cm,0)$)  {0};
		\node[text width=1cm] at ($(gene11.west)+(-0.3cm,0)$) {$O_1$:};
		\draw[-,white] ($(gene10.north)$)-- ++(0,-1.5cm);
		\node (label1) at ($(gene5.south)+(0cm,-0.5cm)$) {(d): Offspring $O_1$ 
			 after angle mutation};
	\end{tikzpicture}
}
\end{center}
\caption{Examples of crossover, length mutation, angle  mutation operator for proposed GA.}
\label{GA:operator}
\end{figure}


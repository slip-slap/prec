\section{Genetic algorithm model}

\input{fig/chapter3_ngam.tex}
\input{fig/chapter3_ogam.tex}
The GA starts off with multiple individuals with limited chromosome lengths, in
which maybe none of these individuals fulfill the constraints. The GA is
assumed to derive appropriate offspring based on the initial population as the
GA continues. The classic way to handle the constrained search of the GA is
either to introduce repair strategies or use a penalty function. Figure
\ref{fig:old_ga_model} shows the classic flow chart of a GA framework which
includes selection, crossover and mutation operators. However, GA is originally
proposed to solve unconstrained problems, therefore, a new approach is
developed to address the constrained GA search problem in an unconstrained way. 

Because of the existence of constraints, the population can be sorted by the
fitness (obtained by the objective function) but can also be sorted by the
constraint value obtained by the constraint function (assuming a constraint
function exists), so the parents of the next generation can be chosen by the
following three approaches. First, the population is sorted by fitness in an
ascending order, and individuals with smaller fitness are selected. These
selected individuals form a group named as proper group. Second, remove
individual which satisfies constraints, and sort population  by the difference
between the individual's constraint value and the threshold of the constraint
in a descending order, and individuals with bigger differences are chosen to be
the parents of next generation. The group which forms are called potential
group, and individual from this group is refered as  potential individual.
Third, the population is sorted by fitness from low to high after removing
individuals which fails to fit the constraints, select individuals with bigger
fitness, and these individuals form the proper group.  So the final parents
pool is consists of three groups, active group, potential group and proper
group.  The number of active individuals, potential individuals and proper
individuals are called, respectively, active number, potential numbers and
proper number. 

Each group in the parents population has its own role in the searching
process.  The problem within traditional GA is premature and has weak local
search ability, therefore, traditional GAs are more likely to get stuck in a
local optimum. To prevent the GA from experiencing early convergence and to
improve the local search performance, the active group is proposed to overcome
this problem. As its name suggests, this group would always live in the
population.  Because both active individual's fitness and constraint value are small,
each individual can be treated as a independent gene clip. So their existance
enriches the gene clip varity of mating pool. The offspring of two active
individuals are more likely to be an active individual, which can maintain the
active group.

For an individual in the potential group, it doesn't satisfy constraint,
however, it's supposed to evolve into a proper individual after multiple
generations by modifying its chromosome structure or length. The crossover
operation could happen between a potential individual with an active
individual, or a potential individual or a proper individual. The child of an
active individual and a potential is more likely to be a potential individual,
and this active individual could inject new gene clip into this potential
individual, therefore providing a new evolution direction. 


An proper individual means a feasible solution, it fulfills all the constraints. 
However, there are still some drawbacks within it, for example, its fitness is low.
The crossover operation  could happen between a proper individual and any other individuals.


Figure \ref{fig:model} shows the flow chart of new GA. First, the
population are divided into three groups, active group, potential group, and
proper group by above mentioned method. Second, select appropriate number of
individuals from each group as parents, and the various selection scheme can be
taken for each group. 


At the beginning of the GA, no individual in the population is appropriate, which means the number
of proper individuals is nearly zero. Therefore, the GA can be divided into two stages according to whether
proper individuals are generated during the search process. During the initial stages, the number of
potential individuals gradually decreases from the maximum (which is the parent population) to the potential
number, while the number of proper individuals increases from zero to the proper number as the GA
continues. After the initial stage, both groups converge to the
potential number and proper number. To differentiate the current selection methods from
the following, the current GA is called the basic GA. In the following experiment, 50 percent of the
parents are potential individuals, and 50 percent of the parents are proper individuals.



\section{Conclusions}
In this paper, we reviewed the use of the proposed GA framework, classical
lamination theory, and Tsai-wu failure theory for the lay-up design for cross
ply laminate. GA is initially designed for unconstrained problem which is not
suitable for a constrained problem. In present study, we deal with this
contrained problem by mix strategies of selection methods instead of adding
extra terms into the objective function. So the constraint problem  can be
solved in an unconstrained way.

This variant of the GA provides a new approach to address the search constraint
in laminate composite optimization, and this method is simple to extend for
solving the multiple constraint search problem in other domains. At the same
time, the proposed GA model is more complicate than the traditional GA model,
which involves more parameters. To advance its performance, the fine-tuning of
those parameters needs more effort. 

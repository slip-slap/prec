\section{Introduction}
Composite materials offer improved strength, stiffness, corrosion resistance,
etc. over conventional materials, and are widely used as alternative materials
for applications in various industries ranging from electronic packaging to golf
clubs, and medical equipment to homebuilding, making aircraft structure to space
vehicles. The stacking sequence and fiber orientation of composite laminates
give the designer additional degree of freedom to tailor the design with
respect to strength or stiffness.  One widely known advantage of using composite
material is can significantly reducing the weight of target structure, and many
researchers attempted to improve the efficiency of using composite material by
minimizing the thickness\cite { schmit1973optimum, schmit1977optimum,
	fukunaga1991strength, soares1995discrete, le1995improved,
	jayatheertha1996application, wang1996optimum, adali1997minimum,
	correia1997higher, scares1997optimization, abu1998optimum, lombardi1998anti,
	le1998design, sivakumar1998optimum, barakat1999use, richard2000reliability,
moita2000sensitivity, soremekun2001composite, walker2003technique,
di2003multiconstrained, kere2003using}.

In practice, fiber orientations are restricted to a finite set of angles and
ply thickness is a specific numeric value.  Because the design variables are
not continueous, a gradient-based optimization procedure, such as the gradient
descent method, is not suitable to cope with such problems.  Moreover, gradient-based
optimization approach is very easily to get trapped in local minima, and
many local optimum may exist in structural optimization problems. A stochastic
optimization, such as the genetic algorithm(GA) and simulated annealing(SA), can
deal with optimization problems with discrete variables. Besides, the stochastic
method could escape from the local optimum, and obtain the global optimum.  GA is one of
the most reliably stochastic algorithms, which has been widely used in solving
constraint design for composite
laminate\cite{callahan1992optimum,soremekun2001composite,park2001stacking,walker2003technique,deka2005multiobjective,pelletier2006multi,jadhav2007parametric,kim2007development,park2008improved}.
Although GA gains different advantages for solving discrete problems, many
disadvantages exist within this approach. First, the optimization process of GA
parameters, such as the population size, parent population,mutation percentage,
etc., is very tedious; Second, the GA needs to evaluate the objective functions
many times to achieve the optimization, and the computation cost is very high;
the last problem within GA is the premature convergence. GA consists of five
basic parts: the variable coding, selection scheme, crossover operator, mutation
operator, and how the constraints are handled.

The first issue when implementing a GA is the representation of design
variables, and an appropriate design representation is crucial to enhance the
efficiency of GA. The canonical GA has always used binary strings to encode
alternative solutions, however, some argued that the minimal cardinality, i.e.,
the binary representation, is not the best option. 

Selection scheme plays a critical role in balancing the dilemma of exploration
and exploitation inherent in GA, and various selection methods, for example,
roulette wheel, elitist, and tournament, etc. have been proposed to overcome
this issue. Both roulette selection and tournament selection are well-studied
and widely employed in the optimization design of laminated composite due to
their simplicity to code and efficiency for both nonparallel and parallel
architectures.

Crossover is another crucial operator introduced into the GA
methodology framework, in which the alternative solution is generated from the
mating pool.  multiple types of crossover operator have been utilized in the optimization
design of composite structures, such as, one-point, two-point, and uniform
crossover.

GA is originally proposed for unconstrained optimization. However, in order to
deal with constrained design for composite laminate, some techniques were
introduced into the GA. The first method is using of data structure, special
data structure was developed to fulfill the symmetry constraint of the laminate,
which consists of coding only half of the laminate and considering that each
stack of the laminate is formed by two laminae with the same orientation but
opposite signs\cite{le1995improved,kogiso1994design}. A penalty function is
developed to convert a constrained problem into an unconstrained problem by
adding a penalty term to the objective function. Another method to solve the
constrained problem is introducing repair strategy by Todoroki and Haftka
\cite{todoroki1998stacking}, which is aim to transform infeasible solutions to
feasible solution by incorporating problem-specific knowledge. 

Another major concern within GA is the convergence speed in terms of the time
and computation cost needed to reach a solution of desired quality. The
objective function based on the CLT is excessively time-consuming and complicate
to evaluate, besides, the target function of GA  needs to calculate many
times. The traditional method to deal with this issue is by increasing the
selection pressure to accelerate the convergence speed, however, in some cases,
this approach does not achieve an ideal result. Because the GAs just provides a
methodological framework to deal with tricky problems, which is heavily
inspired by evolution of biology, it is unnecessary to exactly follow all the
GA operation. It is possible to just perform one or more GA operations, and
incorporate other techniques into GA. In the present study, a variant of mutation
operator is introduced to accelerate the convergence process.
  

To check the feasibility of a laminate composite by imposing a strength
constraint, various failure criterion have been proposed to decide whether it
fails or not, such as  maximum stress failure theory, maximum strain failure
theory, Tsai-Hill Failure theory, and Tsai-Wu criterion. Each theory is proposed
based on massive experiment data or complicate mathematical model, however
single use any of them may lead to a false optimum design for some loading case
due to the particular shape of its failure envelope. In order to overcome this
disadvantage within every failure theory, two reliably failure criteria, maximum
stress theory and Tsai-wu criterion are employed to check whether the composite
laminate fullfills the constraint.

The rest of the paper is organized as follows. Section 2 explains the classical laminate theory and
the failure criteria taken in the present study.  Section 3 explains the proposed method of
selection strategy and self-adaptative parameters for mutation during the GA process. Section 4
describes the result of the numerical experiments in different cases, and in the conclusion section,
we dicuss the results.



The main challenge presented by the laminated composite design is the laminate
layup, involving a set of fiber orientations, composite material systems, and
stacking sequences. In practice, it can be formulate as a constrained
combinatorial optimization problems which can be solved by the genetic
algorithm. However, genetic algorithm is invented for unconstrained problem,
which means to use this algorithm you need convert a constrained problem to an
unconstrained problem, for example, appending punishment items to the objective
function. 

In the present study,  a genetic algorithm methodological framework-based
optimization procedure is proposed for constrained problems 

minimize thickness(or weight) of
midplane-symmetric composite laminate subject to in-plane loading. Fiber
orientation and ply thickness are chosen as design variables, and a variant of GA is 
employed to search for the optimal design of composite laminates.  To avoid
spurious laminate designs, both the Tsai-wu and the maximum stress criteria are
taken to determine whether load-bearing capacity is exceeded or not. Numerical
results are obtained and presented under different loading cases.


In this present study, a new variant of
the genetic algorithm is proposed for the optimal design by modifying the
selection strategy.
To check the feasibility of a laminate subject to in-plane
loading, the effect of the fiber orientation angles and material components on
the first ply failure is studied. Then we compare the experimental results with
works in other literature.

Traditionally, classic lamination theory is widely used to compute
properties of composite materials under in-plane and out-of-plane loading from
a knowledge of the material properties of the individual layers and the
laminate geometry. In this study, a systematic procedure is proposed to design
an artificial neural network for a practical engineering problem, which is
applied to calculate the strength ratio of a laminated composite material under
in-plane loading, in which the genetic algorithm is proposed to optimize the search
process at four different levels: the architecture, parameters, connections of
the neural network, and active functions. 

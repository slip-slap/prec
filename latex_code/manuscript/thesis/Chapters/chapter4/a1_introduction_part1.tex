\section{Introduction}
Fiber-reinforced composite materials have been widely used in a variety of
applications, which include electronic packaging, sports equipment,
homebuilding, medical prosthetic devices, high-performance military
structures, etc. because they offer improved mechanical stiffness, strength,
and low specific gravity of fibers over conventional materials.  The stacking
sequence, ply thickness, and fiber orientation of composite laminates give the
designer an additional ’degree of freedom’ to tailor the design with respect to
strength or stiffness. Classical lamination theory(CLT) and failure theory,
e.g., Tsai-Wu failure criteria, is usually taken to predict the behavior of a
laminate from a knowledge of the composite laminate properties of the
individual layers and the laminate geometry.

However, the use of CLT needs intensive computation which takes an analytical
method to solve the problem, since it involves massive matrix multiplication
and integration calculation. Techniques of function approximation can
accelerate the calculation process and reduce the computation cost.  Artificial
neural network(ANN), heavily inspired by biology and psychology, is a reliable
tool instead of a complicated mathematical model. ANN has been widely used to
solve various practical engineering problems in applications, such as pattern
recognition, nonlinear regression, data mining, clustering,  prediction, etc.
Evolutionary artificial neural networks is a special class of artificial neural
networks, in which evolutionary algorithms are introduced to design the
topology of an ANN, and can be used at four different levels: connection
weights, architectures, input features, and learning rules.  It is shown that
the combinations of ANN's and evolutionary algorithm \cite{lobo2007parameter}
can significantly improve the performance of intelligent systems than that
rely's on ANN's or evolutionary algorithms alone.

The rest of this paper is organized as the following: section II introduces the
CLT and the failure criteria, which is used to check whether the composite
material fails or not in the present study; section III covers the design of
artificial neural network for a function approximation; section IV reviews the
use of the genetic algorithm in the design of neural network architecture, and
the techniques of parameters optimization during the training process; section
V presents the result of the numerical experiments in different cases; in the
conclusion part, we present and discuss the experiment results.





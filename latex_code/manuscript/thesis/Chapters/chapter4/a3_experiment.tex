\section{Experiment}
In the previous section, we present the details of our strategies for designing
an ANN. In this section, we explain the details of the preparation of the
training dataset, and validation dataset.
\subsection{Dataset Preparation}
For composite material, it is impossible to obtain massive training data from
the practical scenario. Therefore, we use classical lamination theory and
failure theory, which follows a two-step procedure: first, evaluate the stress
and strain according to classic lamination theory; second, substitute them into
the corresponding equation to get the strength ratio. We repeat this procedure
to yield 14000 points uniformly distributed over the domain space. We define
the domain of the corresponding inputs as follows: the range of in-plane
loading varies from 0 to 120; the range of fiber orientation $\theta$ is from
-90 to 90; ply thickness $t$ is 1.27mm, the number of plies range $N$ is from 4
to 120. Three different composite material is used in this experiment, as shown
in Tab. \ref{tab:mat}. Fig. \ref{tab:traing-data} shows part of the training
data, which are randomly selected from the generated training dataset.  To
speeds up the learning and accelerate convergence, the input atttributes of the
dataset are rescaled to between 0 and 1.0 by a linear function.

\begin{table*}[!t]	
\centering
\caption{Examples of the training data}
\label{tab:traing-data}
\begin{adjustbox}{width=1\textwidth}
	\begin{tabular}{cccc|cc}
		\toprule
		\multicolumn{4}{c}{\textbf{Input}} &  \multicolumn{2}{c}{\textbf{Output}} \\
		\midrule
		Load  &  \makecell{Laminate \\ Structure }  & \makecell{Material \\ Property} & \makecell{Failure \\  Property}  & MS & Tsai-Wu \\
		\midrule

		-70,-10,-40,  & 90,-90,4,1.27, & 38.6,8.27,0.26,4.14,  & 1062.0,610.0,31,118,72,  & 0.0102, & 0.0086 \\
		-10,10,0,     & -86,86,80,1.27,& 181.0,10.3,0.28,7.17, & 1500.0,1500.0,40,246,68, & 0.4026, & 2.5120 \\
		-70,-50,80,   & -38,38,4,1.27, & 116.6,7.67,0.27,4.173,& 2062.0,1701.0,70,240,105,& 0.0080, & 0.0325 \\
		-70,80,-40,   & 90,-90,48,1.27,& 38.6,8.27,0.26,4.14,  & 1062.0,610.0,31,118,72,  & 0.0218, & 0.1028 \\
		-20,-30,0,    & -86,86,60,1.27,& 181.0,10.3,0.28,7.17, & 1500.0,1500.0,40,246,68, & 0.6481, & 0.9512 \\
		0,-40,0,      & 74,-74,168,1.27,& 181.0,10.3,0.28,7.17,& 1500.0,1500.0,40,246,68, & 1.3110, & 3.9619 \\
		\bottomrule
		\end{tabular}
\end{adjustbox}
\end{table*}


\subsection{ANN training and validation}
The ANN training procedure is carried out by optimising the multinomial
logistic regression objective using mini-batch gradient
descent\cite{lecun1989backpropagation} with momentum. The batch size is set to
1000, momentum to 0.9. the learning rate is set to $10^{-2}$. As for the
training dataset and validation dataset, We follow the 70/30 rule, with 70\% of
the entire data for training and 30\% for validation.

\subsection{Genetic algorithm}
The genetic algorithm involves the evolution of an artificial neural network’s
topology, activation function, etc., in the optimizing process.  The
corresponding parameters are as the following. The population is 10, the
percentage of parents in the population is 40\%; the strategy of selecting
parents is rank-based; the mutation rate of the offspring is 0.3.


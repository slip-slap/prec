\section{Evolutionary Artificial Neural Network}
\begin{figure}[tb]
\centering
	%\resizebox{.7\linewidth}{!}{dd
%\begin{tikzpicture}[show background grid, show background rectangle]
\begin{tikzpicture}
	\useasboundingbox (-3.5,-5.) rectangle (3.5,6.3);
	\scope[transform canvas={scale=1}]
    %\draw[help lines] (-3cm,-7cm) grid (6cm, 5cm);
    \tikzstyle{block} = [rectangle, text centered, draw=black,
    minimum width=1.1cm, minimum height=0.4cm]
	% first level evolution: connection weights
    \node (evaluation-parent) [block, minimum width=2.4cm, minimum
        height=1.8cm,draw=white] {};
    \node (evaluation) [block] at ($(evaluation-parent.north)$) {evaluation};
    \node (reproduction) [block] at ($(evaluation-parent.south)$) {reproduction};
    \node (tasks) [block, minimum width=1.1cm, minimum height=0.4cm] {tasks};

    \draw[->] ($(evaluation.south)+(0.3cm,0cm)$) --
        ($(tasks.north)+(0.3cm,0cm)$) node[auto=left, pos=0.5] {\small weights}; 
    \draw[<-] ($(evaluation.south)+(-0.3cm,0cm)$) --
        ($(tasks.north)+(-0.3cm,0cm)$) node[auto=right, pos=0.5] {\small fitness}; 

	\draw[black] (reproduction.west) -- ++(-0.3cm,0)  |- (evaluation.west);
	\draw[black] (reproduction.east) -- ++(0.3cm,0)   |- (evaluation.east);

    \node (level1) [block,draw=black, minimum width=3.5cm, minimum height=3.0cm] at
        (0cm,0.2cm) {};
    \node [align=left] at ($(level1.north)+(0,-0.2cm)$) {\tiny THE EVOLUTION
        OF};
    \node [align=left] at ($(level1.north)+(0,-0.45cm)$) {\tiny CONNECTION
            WEIGHTS
        };
    % second level evolution: active functions
    \node (level2_container) [block, draw=black, minimum width=5.8cm, minimum
        height=6.0cm] at
        (0, 0.4cm)  {};
    \node [align=left] at ($(level2_container.north)+(0,-0.25cm)$) {\tiny THE EVOLUTION
        OF ACTIVE FUNCTIONS};
    \node (level2-assister) [block, draw=white, minimum width=5cm, minimum
		height=4.6cm] at
        (0, 0.3cm)  {};
    \node (evaluation) [block] at ($(level2-assister.north)$) {\small evaluation of
        active functions};
    \node (reproduction) [block] at ($(level2-assister.south)$) {\small reproduction of
        active functions};

    \draw[->] ($(evaluation.south)+(0.3cm,0cm)$) --
        ($(level1.north)+(0.3cm,0cm)$) node[auto=left, pos=0.5] {\small active functions
        }; 

    \draw[<-] ($(evaluation.south)+(-0.3cm,0cm)$) --
        ($(level1.north)+(-0.3cm,0cm)$) node[auto=right, pos=0.5] {\small fitness}; 

	\draw[black] (reproduction.west) -- ++(-0.3cm,0)  |- (evaluation.west);
	\draw[black] (reproduction.east) -- ++(0.3cm,0)   |- (evaluation.east);
    % third level: evolution of learning rules
    \node (level3-assister) [block, draw=white, minimum width=3cm, minimum
        height=7.6cm] at
        (0, 0.4cm)  {};
	\node (level3_container) [block, draw=black, minimum width=7.5cm, minimum
		height=9cm] at (0, 0.4cm)  {};
    \node [align=left] at ($(level3_container.north)+(0,-0.25cm)$) {\tiny THE EVOLUTION
        OF LEARNING RULES};
    \node (evaluation) [block] at   ($(level3-assister.north)$) {\small evaluation of
        learning rules};
    \node (reproduction) [block] at ($(level3-assister.south)$) {\small reproduction of
        learning rules};
    \draw[->] ($(evaluation.south)+(0.3cm,0cm)$) --
        ($(level2_container.north)+(0.3cm,0cm)$) node[auto=left, pos=0.5] {\small learning
        rule}; 
    \draw[<-] ($(evaluation.south)+(-0.3cm,0cm)$) --
        ($(level2_container.north)+(-0.3cm,0cm)$) node[auto=right, pos=0.5] {\small fitness}; 

	\draw[black] (reproduction.west) -- ++(-0.9cm,0)  |- (evaluation.west);
	\draw[black] (reproduction.east) -- ++(0.9cm,0)   |- (evaluation.east);

	% fourth level: evolution of topology
    \node (level4-assister) [block, draw=white, minimum width=3cm, minimum
        height=11cm] at
        (0, 0.8cm)  {};
	\node (level4_container) [block, draw=black, minimum width=9cm, minimum
		height=11.8cm] at (0, 0.8cm)  {};
    \node (evaluation) [block] at   ($(level4-assister.north)$) {\small evaluation of
        topology};
    \node (reproduction) [block] at ($(level4-assister.south)$) {\small reproduction of
        topology};
    \draw[->] ($(evaluation.south)+(0.3cm,0cm)$) --
        ($(level3_container.north)+(0.3cm,0cm)$) node[auto=left, pos=0.5] {\small learning
        topology}; 
    \draw[<-] ($(evaluation.south)+(-0.3cm,0cm)$) --
        ($(level3_container.north)+(-0.3cm,0cm)$) node[auto=right, pos=0.5] {\small fitness}; 
	\draw[black] (reproduction.west) -- ++(-2.1cm,0)  |- (evaluation.west);
	\draw[black] (reproduction.east) -- ++(2.1cm,0)   |- (evaluation.east);

    %\node (evaluation) [block] at ($(level3-assister.north)$) {\small evaluation of
    %    architecture};
    %\node (reproduction) [block] at ($(level3-assister.south)$) {\small reproduction of
    %    learning architecture};
    % level 6
    %\node (level6) [block, minimum width=7.0cm, minimum
    %    height=8.1cm] at
    %    (0, 0.5cm)  {};
    %\node [align=left] at ($(level6.north)+(0,-0.2cm)$) {\tiny THE EVOLUTION OF
    %        ARCHITECTURE
    %    };
	\endscope
\end{tikzpicture}
%}
\caption{A general framwwork for EANN, in which fitness refers to the corresponding value of objective function.}
\label{fig:evolution}
\end{figure}


\subsection{General neural network}
In this paper, the feedforward ANN is adopted in the current
study, since it is straightforward and simple to code. For function
approximation through an ANN, Cybenko demonstrated that a two-layer perceptron
can form an arbitrarily close approximation to any continuous nonlinear
mapping\cite{cybenko1989approximation}. Therefore, a two-layer feedforward ANN
is proposed in the present study. Fig. \ref{fig:gnn} shows a general framework for a
two-layer NN, in which the number of nodes in the hidden layer and the
connection with inputs, are critical in the design of an ANN. For nodes in the
hidden layer, we can think of them as feature extractors or detectors.
Therefore, nodes within it should partially be connected with the inputs of an ANN,
since the unnecessary connections would increase the model's complicacy, which
will reduce an ANN’s performance. Because we treat the nodes in the hidden layer
as feature extractors, so the number of nodes in this layer should be less than
the number of inputs. For the nodes in the last layer, every node should be
fully connected with nodes in the previous layer, since we think of the nodes
in the hidden layer as features. The rest, which affects a NN’s performance,
are activation function, and ANN's training method. In the following section, we
denote the $i$th node in the input layer, and the hidden layer, as $i_i$, and $h_i$,
respectively.

\begin{figure}
	\centering
	\begin{tikzpicture}
		[ plain/.style={ draw=none, fill=none, }, remember picture, net/.style={ matrix of nodes, nodes={ draw, circle,
			inner sep=7.5pt
			},
		  nodes in empty cells,
		  column sep=-10.5pt,
		  row sep=0.8cm
		  }
		]
	%\draw[help lines] (-3cm,-6cm) grid (6cm,3cm);
	\matrix[net] (mat)
		{
				  & |[plain]| &           & |[plain]|  &           & |[plain]| &           &  |[plain]|      &               \\
		|[plain]| &           & |[plain]| &            & |[plain]| &           & |[plain]| &                 & |[plain]|     \\ 
		|[plain]| & |[plain]| &           & |[plain]|  &           & |[plain]| & 	  	   &  |[plain]|      & |[plain]|     \\ 
	  };

	  \node at ($(mat-1-1.west)+(-16pt,0)$) {Input: };
	  \node at ($(mat-2-2.west)+(-24pt,0)$) {Hidden:};
	  \node at ($(mat-3-2.west)+(-24pt,0)$) {Output:};
	  \node at (mat-1-1.base) {$i_1$};
	  \node at (mat-1-3.base) {$i_2$};
	  \node at (mat-1-5.base) {...};
	  \node at (mat-1-7.base) {$i_{n-1}$};
	  \node at (mat-1-9.base) {$i_{n}$};
	  \node at (mat-2-2.base) {$h_1$};
	  \node at (mat-2-4.base) {$h_2$};
	  \node at (mat-2-6.base) {$...$};
	  \node at (mat-2-8.base) {$h_{m}$};
	  \node at (mat-3-5.base) {$...$};

		 \foreach \a in {1,3}{
			\foreach \b in {2,6}{
				\draw[->] (mat-1-\a.south) -- (mat-2-\b.north);
			 }
		  }
		 \foreach \a in {3,7,9}{
			\foreach \b in {4,8}{
				\draw[->] (mat-1-\a.south) -- (mat-2-\b.north);
			 }
		  }

		 \foreach \c in {2,4,6,8}{
			\foreach \d in {3,5,7}{
				\draw[->] (mat-2-\c.south) -- (mat-3-\d.north);
			}
	 }
\end{tikzpicture}
\caption{General Neural Network}
\label{fig:gnn}
\end{figure}





\subsection{Activation function}

The activation function is one of the critical parts of an ANN. Liu
\cite{liu1996evolutionary} et al. claims that the performance of neural networks with
different activation functions is different, even if they have the same
architecture.  A generalized activation function can be written as

\begin{equation}
	y_i = f_i(\sum_{j=1}^n{w_{ij}x_j - \theta})
\end{equation}

where $y_i$ is the output of the node $i$, $x_j$ is the $j$th input to the
node, and $w_{ij}$ is the connection weight between adjacent nodes $i$ and $j$.
Tab. \ref{tab:transfer_function} displays the most widely adopted activation
functions in the design of an ANN, which is used for the current study.

\begin{table*}[!t]
\centering
\caption{Different Activation Functions}
\label{tab:active_function}
\begin{adjustbox}{width=1\textwidth}
\label{tab:transfer_function}
	\begin{tabular}{lllc}
			\toprule
			Type & Description  & Formula & Range  \\
			\midrule
			Linear   & The output is proportional to the input & $f(x)=cx$                  &  $(-\infty, +\infty)$ \\
			Sigmoid  & A family of S-shaped functions          & $f(x)=\frac{1}{1+e^{-cx}}$ & $(0, 1)$ \\
			tanh     & A family of Hyperbloic functions        & $f(x)=\frac{e^x -e^{-x}}{e^x+e^{-x}}$ & $(0, 1)$ \\
			Gaussian & A coninuous bell-shaped curve           & $f(x)=e^{-x^2}$            & $(0,1)$ \\ 
			ReLU     & A piece-wise function                   & $f(x)= max{0,x}$           & $(0, +\infty)$ \\
			Softplus & A family of S-shaped functions          & $f(x) = ln(1+e^x)$         & $(0, +\infty)$ \\
			\bottomrule
	\end{tabular}
\end{adjustbox}
\end{table*}


\subsection{Weights learning}
The weight training in an ANN is to minimize the error function, such as the
most widely used mean square error function, which calculates the difference
between the desired and the prediction output values averaged overall examples.
Gradient descent algorithm is widely adopted to reduce the value of an error
function, which has been successfully applied in many practical areas. However,
this class of algorithms is plagued by the possible existence of local minima
or ”flat spots” and ”the curse of dimensionality.” One method to overcome this
problem is to adopt a genetic algorithm(GA)


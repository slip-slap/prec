% Chapter Template

\chapter{Introduction} % Main chapter title

\label{Chapter1} % Change X to a consecutive number; for referencing this chapter elsewhere, use \ref{ChapterX}

%----------------------------------------------------------------------------------------
%	SECTION 1
%----------------------------------------------------------------------------------------

\section{Laminated Composite Material}

Composite materials offer improved strength, stiffness, corrosion resistance,
etc. over conventional materials, and are widely used as alternative materials
for applications in various industries ranging from electronic packaging to golf
clubs, and medical equipment to homebuilding, making aircraft structure to space
vechicles. The stacking sequence and fiber orientation of composite laminates
give the designer additional 'degree of freedom' to tailor the design with
respect to strength or stiffness.  One widely known advange of using composite
material is can significantly reducing the weight of target structure, and many
researchers attempted to improve the efficiency of using composite materail by
mimimizing the thickness.

\section{Classic Lamination Theory and Failure Theory}
Classic lamination theory(CLT) is used to develop the stress-strain relationship of
composite material under in-plane and out-of-plane loading. First, develop
stress-strain relationships, elastic moduli, strengths of an angle ply based on
a unidirectional lamina and the angle of the ply; second, becasue a laminate is
consist of more than one lamina bonded together through their thickness, so the
macromechianical analysis will be developed for a laminate based on applied
loading.  To check whether a designed lay-up is plausible or not, different
failure theories have been developed.

\section{Genetic Algorithm}
In the design of composite material, gradient based optimization techiques are
not appliable in this domain, because the design varaibles, such as fiber
orientation, layer thickness, number of layers etc. are discrete. Genetic
algorithm(GA) can be adopted in the optimization problem becasue it doesn't require
the gradient information. Moreover, the GA has been proved a reliable techique
and widely used in the design of composite material. 

\section{Artificial Neural Network}
CLT is an classic analytical approach to obtain the stress and stain of
composite material, the disadvantage of this method is quiet cubersome and in
which involves compliate matrix and integration operations. Artificial neural
network(ANN) has been proved a reliable tool in modelling various engineering
system in practice without solving tricky equations and making ideal
assumptions. In this thesis, the ANN is taken to approximate the numberic
results based on CLT and failure theory.

\section{Summary}
In this thesis, first, we review the use of composite material in practice,
then, the CLT to calculate the stress and strain under certain loading, last,
the failure theories which are used to decide whether a composite material will
failure or not. Second, the stochastic algorith, GA, is studied and implemented
in the desgin of composite material, two different cases are studied in which GA
can be taken to obtain the optimal lay-up. At last, ANN is introduced to
approximate the evaluation result of CLT, the reason for adopting ANN is to
reduce the computation complexity based on CLT.





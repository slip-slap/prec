% Chapter Template

\chapter{A New Genetic Algorithm Model for Constrained Problem} % Main chapter title

\label{Chapter3} % Change X to a consecutive number; for referencing this chapter elsewhere, use \ref{ChapterX}

%----------------------------------------------------------------------------------------
%	SECTION 1
%----------------------------------------------------------------------------------------

\section{Genetic Algorithm Framework}
GA is one of the most reliable stochastic algorithm, which has been widely used
in discrete variables optimization problems.  
\section{Constrained Problem Optimization}

%----------------------------------------------------------------------------------------
%	SECTION 2
%----------------------------------------------------------------------------------------
\section{A New Genetic Algorithm Model}

\subsection{Encoding}
Due to the simplicity and efficiency of float representation, this encoding
method is implemented to represent a possible solution. As shown in Figure \ref{GA:operator}
 (a), these two chromsomes represent a $[+8_{7}/-9_{2}]_s$
carbon T300/5308 laminated composite, and $[+19_{4}/-36_{6}]_s$, respectively.
Becasue the laminate adopted in this paper is symmetric to its mid-plane, so
only half needs to be encoded.

\subsection{Selection}
The purpose of the selection operator is to chose mating pool to produce
alternative solutions of better fitness. Traditional methods of selecting
strategies only take the fitness of individuals into acount, however, due to 
the existance of constraint, various selection schemes are implemented to
selecet the mating set. Based on different selection schemes, the parents of
next generation can be divided into  three groups: proper groups, active groups,
and potential groups according to different selecting methods. 

Proper parents mean in which individual fullfils the constraints, which are
chosen by the individual's fitnees, individuals with better fitness are more
likely to be chosen if they fit the constraint; active groups means that
individual is supposed to be always exist in the parents during the GA, which
are selected by fitness, ignoring the constraint; The individuals from active
group may not correspond to feasible solutions, but their existance enriches the
variety of the gene clips.  Potential groups means that they are likely to turn
into proper individual after a couple of generations, and potential individuals
are chosen by constraint function, the more the individual fulfils the
constraint, the more possiblity it will be selected.

\subsection{Crossover}
The crossover operator happens among these three groups. the child of two proper
groups are more likely to be a proper individual which can be used to obtain a
alternative feasible solution. the child of an active individual and a potential
individual can significantly change the gene of active individual's chromsome,
which makes the individual evolve toward a new direction. The offspring of two
active individuals are more likely to be an active individual, which can maitain
the active group.  The figure.\ref{GA:operator} (b) shows two children $O_1$



\section{Summary}


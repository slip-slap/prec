% Chapter Template

\chapter{A New Genetic Algorithm Model for Constrained Problem} % Main chapter title

\label{Chapter3} % Change X to a consecutive number; for referencing this chapter elsewhere, use \ref{ChapterX}

%----------------------------------------------------------------------------------------
%	SECTION 1
%----------------------------------------------------------------------------------------

\section{Genetic Algorithm Framework}
GA is one of the most reliable stochastic algorithm, which has been widely used
in discrete variables optimization problems  \cite { schmit1973optimum,
	schmit1977optimum, fukunaga1991strength, soares1995discrete, le1995improved,
	jayatheertha1996application, wang1996optimum, adali1997minimum,
	correia1997higher, scares1997optimization, abu1998optimum, lombardi1998anti,
le1998design, sivakumar1998optimum, barakat1999use, richard2000reliability,
moita2000sensitivity, soremekun2001composite, walker2003technique,
di2003multiconstrained, kere2003using}. In practice, fiber orientations are
restricted to a finite set of angles, and ply thickness is a specific numeric
value. Because the design variables are not continueous, a gradient based
optimization procedure, such as gradient descent method, is not suitable to cope
with such problems. Moreover, gradient based optimization approach is very eazy
to get trapped in local minima, and many local optimum may exist in structural
optimization problems. A stochastic algorithm, such as GA, is able to deal with
optimization problems with discrete design variables. Besides, stochastic method
could escape from local optimum, and obtain global optimum. 

Although GA gains different advantages for solving discrete problems, many
disadvantages exists within this method. First, the optimization process of GA
parameters, such as the population size, parent population, mutation percentage,
etc., is very tedious; Second, the GA needs to evaluate the objective function
many times to acheive the optimization, and the computation cost is very high;
the last problem within GA is the premature convergence. GA consists of five
basic parts: the variable encoding method, selection scheme, crossover operator,
mutation operator, and how the constraints are handled. A typical GA process is
show in Figure \ref{GA:old_model}.

\begin{figure}
\centering
	\begin{tikzpicture}[thick, scale=1.2, every node/.style={transform shape}]
		\tikzstyle{startstop} = [rectangle, rounded corners, minimum width=1.0cm,minimum height=0.6cm, text centered, draw=black]
		\tikzstyle{io} = [trapezium, trapezium left angle=70, trapezium right angle=110, minimum width=2cm, minimum height=0.6cm, text centered, draw=black]
		\tikzstyle{process} = [rectangle, minimum width=2cm, minimum height=0.6cm, text centered, draw=black]
		\tikzstyle{decision} = [diamond,minimum width=2cm, minimum height=1.2cm, draw=black]
		\node (fitness) [process] {Evaluate all individuals};
		\node[yshift=-0.5cm] (decision) [decision, below of=fitness] {} node at (decision.base) {Is Converge?};
		\node[yshift=-0.5cm] (selection) [process, below of=decision] {Selection};
		\node (crossover) [process] at ($(selection.south)+(0,-0.8cm)$) {Crossover};
		\node (mutation) [process] at ($(crossover.south)+(0,-0.8cm)$)  {Mutation};
		\node (end) [startstop] at ($(selection.west)+(-0.8cm,0cm)$) {End};

		\draw [->] (fitness) -- (decision);
		\draw [->] (decision.south) -- (selection.north) node[auto=left,pos=0.5]{No};
		\node at ($(decision.west)+(-0.4cm, 0.1cm)$) {Yes};
		\draw [->] (selection.south) -- (crossover.north);
		\draw [->] (crossover.south) -- (mutation.north);
		\draw [->] (decision.west) -| (end.north);
		\draw [->] (mutation.east) -- ($(mutation.east)+(1.2cm,0cm)$) |-
			(fitness.east);
	\end{tikzpicture}
\caption{Traditional GA Model.}
\label{fig:old_ga_model}
\end{figure}


The first issue when implementing a GA is the representation of design
variables, and an appropriate design representation is crucial to enhance the
efficiency of GA. The canonical GA has always used binary strings to encode
alternative solutions, however, some argued that the minimal cardinality, i.e.,
the binary representation, are not the best option. Real value string has been
widely employed in 

Selection scheme plays a critical role in balancing the dilemma of exploration
and exploitation inherented in GA, and various selection methods, for example,
roulette wheel, elitist, and tournament etc., have been proposed to overcome
this issue. Both of roulette selection and tournament selection are well-studied
and widely employed in the optimization design of laminated composite due to
their simplicity to code and efficiency for both nonparallel and parallel
architectures.

Crossover is another crucial operator introduced into the GA
methodology framework, in which the alternative solution is generated from the
mating pool.  multiple types of crossover operator has been utilized in the optimization
design of composite structures, such as: one-point, two-point, and uniform
crossover.
.

\section{Constrained Problem Optimization}
However, GA is originally proposed for unconstrained optimization problem, in
order to deal with constrained design for composite laminate, some techniques
are introduced into GA. The first method is using of data structure, special
data structure has been developed to fulfils the corresponding constraint, for
example, in order to fulfill the symmetry constraint of a laminate, the
chromsome is consist of coding only half of the laminate and considering that
each stack of the laminate is formed by two laminate with the same orientation
but opposite signs \cite{le1995improved,kogiso1994design}. The second approach
is reformulating the objective function.  A penalty function is developed to
convert a constrained problem into an unconstrained problem by adding penalty
term to the objective funtion. Another method to solve constrained problem is
introducing repair strategy by Todoroki and Haftka \cite{todoroki1998stacking},
which is aim to transform infeasible solutions to feasible solution by
incorporating problem-specific knowledge. 



%----------------------------------------------------------------------------------------
%	SECTION 2
%----------------------------------------------------------------------------------------
\section{A New Genetic Algorithm Model}

\begin{figure}
\begin{center}
	\begin{tikzpicture}[thick, scale=0.6, every node/.style={transform shape}]
	\tikzstyle{rec} = [rectangle, minimum width=4cm, minimum height=0.8cm,
	text centered, draw=black]
	\tikzstyle{subgroup} = [rectangle, minimum width=1.5cm, minimum height=0.6cm,
	text centered, draw=black]
	\tikzstyle{bigsubgroup} = [rectangle, minimum width=2.5cm, minimum height=0.6cm,
	text centered, draw=black]
	% population
	\node (population) [rec] {population};
	% active group
	\node (active_group_1) at ($(population.south)+(-3cm, -2.0cm)$) [subgroup]
		{active group};
	\node (active_group_2) at ($(active_group_1.south)+(0cm, -2.0cm)$)
		[subgroup] {active parents};
	\draw[->] (population.south) -- (active_group_1.north);
	\draw[->] (active_group_1.south) -- (active_group_2.north);
	% potential group
	\node (potential_group_1) at ($(population.south)+(0cm, -2.0cm)$) [subgroup]
		{potential group};
	\node (potential_group_2) at ($(potential_group_1.south)+(0cm, -2.0cm)$)
		[subgroup] {potential parents};
	\draw[->] (population.south) -- (potential_group_1.north);
	\draw[->] (potential_group_1.south) -- (potential_group_2.north);
	\node at ($(potential_group_1.north)+(0cm, 1.0cm)$) {classification };
	\node at ($(potential_group_2.north)+(0cm, 1.0cm)$) {selection};
    % proper group
	\node (proper_group_1) at ($(population.south)+(3cm, -2.0cm)$) [subgroup]
		{proper group};
	\node (proper_group_2) at ($(proper_group_1.south)+(0cm, -2.0cm)$)[subgroup]
		{proper parents};
	\draw[->] (population.south) -- (proper_group_1.north);
	\draw[->] (proper_group_1.south) -- (proper_group_2.north);
	% crossover
	\node (after_cross_over) at ($(potential_group_2.south)+(0cm, -2.0cm)$) [rec] {};
	\node  at ($(after_cross_over.north)+(0cm, 1.0cm)$)  {crossover};
	\draw[-] ($(after_cross_over.south)+(-0.5cm,0cm)$) --
		($(after_cross_over.north)+(-0.5cm,0cm)$);
	\draw[->] (active_group_2.south) -- (after_cross_over.north);
	\draw[->] (potential_group_2.south) -- (after_cross_over.north);
	\draw[->] (proper_group_2.south) -- (after_cross_over.north);
	% mutation
	\node (active_group_3) at ($(after_cross_over.south)+(-2.2cm, -2.0cm)$)
		[subgroup] {active offspring};
	\node at ($(after_cross_over.south)+(0cm, -1.0cm)$) {mutation};
	\draw[->] ($(after_cross_over.south)+(-1cm,0cm)$)--(active_group_3.north);
	\node (poteni_and_prop) at ($(after_cross_over.south)+(2.2cm, -2.0cm)$)
		[bigsubgroup] {potential and proper offspring};
	\draw[->] ($(after_cross_over.south)+(1cm,0cm)$)--(poteni_and_prop.north);

	% final draw
	\draw[->] (poteni_and_prop.east) --($(poteni_and_prop.east) + (0.2cm,0cm)$) |- (population.east);
	\draw[->] (active_group_3.west) -- ($(active_group_3.west) + (-1cm,0cm)$)
		|- (population.west);
\end{tikzpicture}
\end{center}
\caption{GA Model\label{GA:model}}
\end{figure}


\subsection{Classification}
The population is randomly generated, for every individual in this population
the corresponding constraint numberic value can be obtained. As shown in
Figure \ref{GA:model},The population can
be divided into three groups according to constraint value, which are active
group, potential group and proper group

\subsection{Selection}
The purpose of the selection operator is to chose mating pool to produce
alternative solutions of better fitness. Traditional methods of selecting
strategies only take the fitness of individuals into acount, however, due to 
the existance of constraint, various selection schemes are implemented to
selecet the mating set. Based on different selection schemes, the parents of
next generation can be divided into  three groups: proper groups, active groups,
and potential groups according to different selecting methods. 

Proper parents mean in which individual fullfils the constraints, which are
chosen by the individual's fitnees, individuals with better fitness are more
likely to be chosen if they fit the constraint; active groups means that
individual is supposed to be always exist in the parents during the GA, which
are selected by fitness, ignoring the constraint; The individuals from active
group may not correspond to feasible solutions, but their existance enriches the
variety of the gene clips.  Potential groups means that they are likely to turn
into proper individual after a couple of generations, and potential individuals
are chosen by constraint function, the more the individual fulfils the
constraint, the more possiblity it will be selected.

\subsection{Crossover}
The crossover operator happens among these three groups. the child of two proper
groups are more likely to be a proper individual which can be used to obtain a
alternative feasible solution. the child of an active individual and a potential
individual can significantly change the gene of active individual's chromsome,
which makes the individual evolve toward a new direction. The offspring of two
active individuals are more likely to be an active individual, which can maitain
the active group.  The figure.\ref{GA:operator} (b) shows two children $O_1$

\subsection{Mutation}
i


\section{Summary}
In this chapter, frist, we review the application of tradional GA in the design
of composite material; then A new GA framework has been come up with, in the
following chapters, this NGAM will be adopted to directed the lay-up design of
composite material.


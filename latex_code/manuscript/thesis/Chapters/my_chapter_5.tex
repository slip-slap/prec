% Chapter Template

\chapter{Conclusion} % Main chapter title

\label{Chapter5} % Change X to a consecutive number; for referencing this chapter elsewhere, use \ref{ChapterX}

In this work, it was demonstrated that the application of genetic algorithm and
artificial neural network in the area of composite material. First, we implement
the re-designed GA method which achieved better performance compared with the
results in related literature. Second,  It has shown artificial neural network
can be treated as an alternative method to evaluate the strength ratio of
composite material under in-plane loading, instead use of classical lamination
theory and failure theory.


In this thesis, first, we review the use of composite material in practice,
then, the CLT to calculate the stress and strain under certain loading, last,
the failure theories which are used to decide whether a composite material will
failure or not. Second, the stochastic algorith, GA, is studied and implemented
in the desgin of composite material, two different cases are studied in which GA
can be taken to obtain the optimal lay-up. At last, ANN is introduced to
approximate the evaluation result of CLT, the reason for adopting ANN is to
reduce the computation complexity based on CLT.

This variant of the GA provides a new approach to address the constrained
search for optimization of laminated composite, and this method can be easy
to apply in other domains. At the same time, the proposed GA model is more
complicated than the traditional GA model, which involves more parameters. To
advance its performance, the fine-tuning of those parameters need more effort. 

There are more improvements we can make over the search strategy and
application in the area of laminated composite material. The future work is to
develop a more sophisticated ANN, which not only can predict the properties for
angle ply laminate, but also the other type of laminated composite material.
